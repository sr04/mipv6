\subsection{DHCPv6}

Le DHCPv6 est l’équivalent du procole DHCP de l'IPv4, mais adapté pour l'IPv6.

On peut le voir comme une extension de DHCPv4 puisque les deux sont basés sur le protocole BOOTP. A la différence de DHCPv4 qui a des fonctionnalités limitées, DHCPv6 permet la remise de paramètres de configuration d’un serveur DHCP à un client IPv6. 

C’est un protocole de configuration dynamique.
Il permet d’attribuer les adresses IPv6. 
DHCPv6 est utile si l’on souhaite centraliser la distribution d’IP. 
Par exemple, à l’UTC, on souhaite toujours donner la même IP à un étudiant, peu importe l’endroit où il se trouve. 

\subsubsection{Modes d’opération}

DHCPv6 opére dans plusieurs modes :
\begin{itemize}
  \item stateless mode: fourniture de serveurs DNS et d’autres services commes les options pour les téléphones SIP
  \item stateful mode : une machine peut configurer son adresse IP avec DHCPv6
  \item DHCPv6-PD: obtention d’un préfixe réseau par un routeur à partir d’un fournisseur de service
\end{itemize}

\subsubsection{Mécanisme de communication}

\scalefig{images/dhcpv6_1.png}{.7}{Communication client-serveur DHCPv6}

Nous avons les différents types de flags suivants : 
\begin{itemize}
  \item o : indique qu’il y a d’autres informations à trouver dans le serveur 
  \item m: mode géré c’est à dire que le client doit demander une adresse IP et ne pas en configurer une de manière statique (statelessly)
\end{itemize}

\subsubsection{Identification des clients par le serveur}

DHCPv6 apporte une nouveauté concernant l’identification des machines. 
Contrairement à IPv4 où on identifie la machine avec l’adresse MAC (ARP), DCHPv6 utilise le DUID (Device Unique Identifier).
Chaque participant DHCP en a un, aussi bien les clients que les serveurs. 
C’est celui-ci qui les identifie tout au long des processus de communication.

\subsubsection{Messages DHCPv6}

Chaque unité d’information du protocole DHCPv6 est représentée par un message. 
DHCPv6 reprend en fait le mode de fonctionnement du segment TCP: celui-ci a la même structure pour toutes les fonctions de TCP. Ainsi, les messages DHCPv6 ont un format d’en-tête identique. 

\scalefig{images/dhcpv6_2.png}{.5}{Structure des messages DHCPv6}

\begin{itemize}
	\item msg-type + transaction ID : (codée sur 32 bits) cette partie désigne la fonction de DHCPv6 concernée par l’échange. msg-type désigne le type de message et transaction ID, l’identifiant de la transaction ou de l’échange. msg-type est codé sur 8 bits. 

	\item adresse du serveur : indique l’adresse IPv6 du serveur lors d’un échange DHCP. Cette adresse contient l’adresse de l’interface utilisé par le serveur dans la transaction.

	\item options : sert à faire véhiculer des informations de configurations
\end{itemize}


\subsubsection{Les types de message}

Les types de messages du protocole DHCPv6 sont au nombre de 12. Les voici: 

\begin{itemize}
	\item Sollicitation DHCP (DHCP Solicit) : C’est le message d’interrogation de présence de serveurs DHCP. Un client émet un tel message pour localiser les serveurs DHCP 
	\item Annonce DHCP (DHCP Advertise) : c’est un message servant à désigner la présence de serveurs DHCP. Il est émis après émission d’un message de sollicitation de la part d’un client. 
	\item Requête DHCP (DHCP Request) : c’est un message envoyé par le client (sans adresse) pour demander des paramètres de configurations
	\item Confirmation DHCP (DHCP Confirm) : c’est un message qui sert à vérifier la validité des paramètres octroyés au client
	\item  DHCP (DHCP Renew) : comme son nom l’indique, c’est un message de demande de renouvellement (ou de prolongation) de l’adresse IP affectée 
	\item Ré affectation DHCP (DHCP Rebind) : identique au précédent message à la différence que cette fois-ci la réponse peut venir d’un autre serveur (différent de celui initiateur de l’allocation d’adresse IP)
	\item Réponse DHCP (DHCP Reply) : c’est un message émis par le serveur après avoir reçu une demande du client. C’est ce message qui contient les paramètres de configuration demandés par le client
	\item Libération DHCP (DHCP Release) : c’est un message qui indique que le client a libéré une pile d’adresses IP préalablement allouées par le serveur 
	\item Refus DHCP (DHCP Decline) : c’est un message qui indique qu’une ou plusieurs adresses IP sont déjà occupées 
	\item Notification de reconfiguration DHCP (DHCP Reconfigure-init) : c’est un message envoyé par le serveur au client pour lui indiquer que ses paramètres de configuration ont été modifiés, contenant donc de nouvelles valeurs 
	\item Encapsulation relais (Relay-Forward) : c’est le message dans lequel sont encapsulés les messages envoyés par le client au serveur
	\item Encapsulation serveur (Relay-Reply) : c’est un message généré par le serveur . Ce message sera ensuite extrait par le relais pour le transmettre sur le lien du client
\end{itemize}

\subsubsection{Options}

Précédemment, nous avons parlé d’options qui contiennent les paramètres de configuration des échanges.  
Elles se distinguent en : 

\begin{itemize}
	\item les informations temporaires
	\item  informations générales
	\item les informations spécifiques à DHCP
\end{itemize}

Comme exemple d’options, nous avons :

\begin{itemize} 
	\item l’association d’identification (liste des adresses IPv6 d’une interface du client)
	\item serveur de nom de domaine : liste des serveurs DNS autorisés par le client 
	\item identifiant DHCP : DUID ou identifiant permanent du client
\end{itemize}



