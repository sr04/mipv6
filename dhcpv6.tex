\section{DHCPv6}

Le DHCPv6 est l’équivalent du procole DHCP de l'IPv4, mais adapté pour l'IPv6.
C’est un protocole de configuration dynamique.
Il permet d’attribuer les adresses IPv6. 

Il a été mis en place car IPv6 voulait se débarrasser de la dépendance avec DHCP pour configurer les adresses. 
Ainsi, les hôtes peuvent maintenant configurer plusieurs adresses eux-mêmes. 

En plus d’être un protocole de configuration, il joue aussi le rôle de mécanisme d’auto configuration au même titre que NDP. 

DHCPv6 opére dans plusieurs modes :
\begin{itemize}
  \item stateless mode: fourniture de serveurs DNS et d’autres services commes les options pour les téléphones SIP
  \item stateful mode : une machine peut configurer son adresse IP avec DHCPv6
  \item DHCPv6-PD: obtention d’un préfixe réseau par un routeur à partir d’un fournisseur de service
\end{itemize}

\scalefig{images/dhcpv6_1.png}{.7}{Communication client-serveur DHCPv6}

Nous avons les différents types de flags suivants : 
\begin{itemize}
  \item o : indique qu’il y a d’autres informations à trouver dans le serveur 
  \item m: mode géré c’est à dire que le client doit demander une adresse IP et ne pas en configurer une de manière statique (statelessly)
\end{itemize}

Il est aussi à noter que les machines s’identifient au serveur DHCP avec le DUID contrairement à IPv4 où on se connecte avec l’adresse MAC. 
DUID signifie Device Unique IDentifier (identifiant unique de l’appareil). Et chaque participant DHCP en a un. C’est celui-ci qui l’identifie tout au long de la communication.

Enfin, pour communiquer, des messages DHCP sont envoyés; voici  la structure de ces messages: 

\scalefig{images/dhcpv6_2.png}{.5}{Structure des messages DHCPv6}

