\section{Conclusion}

Cette étude nous a permis d'acquérir de bonnes connaissances de base sur IPv6, qui va devenir très rapidement omniprésent.
Mobile IPv6 est une fonctionnalité supplémentaire apportée par IPv6.
Elle montre que le protocole permet une construction modulaire : la sécurité de Mobile IPv6 peut ainsi être implémentée de différentes manières, tout en conservant un protocole de base compatible.

Cependant, cette modularité est aussi une faiblesse : Cisco et UMIP n'ont pas choisi d'implémenter les mêmes techniques pour la sécurisation, ce qui pose des soucis d'interopérabilité.
Cette modularité fait penser à l'authentification sur les réseaux Wi-Fi, que nous avons étudiée en SR06.
La RFC5778\footnote{\url{http://tools.ietf.org/html/rfc5778}} décrit d'ailleurs une méthode pour s'authentifier avec de l'EAP en Mobile IPv6, un protocole qui est couramment utilisé sur les réseaux Wi-Fi aujourd'hui.

Enfin, 10 ans après la publication de la première RFC sur Mobile IPv6, son support reste relativement limité : une des implémentations sous Linux a été abandonnée et l'utilisation sous les dernières variantes de Debian nécessite une compilation de noyau.
De plus, le processus nécessite une configuration manuelle qui réserve pour l'instant son utilisation à des cas très précis.

Mobile IPv6 nécessite donc encore quelques développements avant d'être utilisable à grande échelle, mais apporte une fonctionnalité précieuse à IPv6.
Le nécessaires sera donc très probablement réalisé avec le déploiement d'IPv6 dans les années qui viennent.


%                                                               %
%                        \                                      %
%                         \                                     %
%                          \\                                   %
%                           \\                                  %
%                            >\/7                               %
%                        _.-(6'  \                              %
%                       (=___._/` \                             %
%                            )  \ |                             %
%                           /   / |                             %
%                          /    > /                             %
%                         j    < _\                             %
%                     _.-' :      ``.                           %
%                     \ r=._\        `.                         %
%                    <`\\_  \         .`-.                      %
%                     \ r-7  `-. ._  ' .  `\                    %
%                      \`,      `-.`7  7)   )                   %
%                       \/         \|  \'  / `-._               %
%                                  ||    .'                     %
%                                   \\  (                       %
%                                    >\  >                      %
%                                ,.-' >.'                       %
%                               <.'_.''                         %
%                                 <'                            %
%                                                               %