\section{Introduction}

IPv6 est la nouvelle version du protocole IP (couche 3 du modèle OSI), succédant à IPv4, qui est aujourd'hui couramment utilisée.
Elle a été développée pour plusieurs raisons.
\begin{itemize}
  \item À cause de la pénurie d’adresses au niveau mondial, il fallait pouvoir proposer une solution permettant d'augmenter l'espace d'adressage.
  \item Plusieurs champs de l'en-tête IPv4 sont peu ou pas du tout utilisé. IPv6 apporte une simplification du format de l'en-tête.
  \item Les changements de l'en-tête permettent une gestion plus flexible des options et permet l'ajout de nouvelles options dans le futur.
  \item Des fonctionnalités d'authentification et de confidentialité auparavant séparées du protocole IP, sont maintenant directement implémentées dans le protocole.
\end{itemize}
  
Cette étude présente les caractéristiques de cette nouvelle version, puis quelques détails sur son implémentation.
  
