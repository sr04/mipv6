\documentclass[a4paper,11pt,final]{article}
% Pour une impression recto verso, utilisez plutôt ce documentclass :
%\documentclass[a4paper,11pt,twoside,final]{article}

\usepackage[english,francais]{babel}
\usepackage[utf8]{inputenc}
\usepackage[T1]{fontenc}
\usepackage[pdftex]{graphicx}
\usepackage{setspace}
\usepackage{hyperref}
\usepackage[french]{varioref}
\usepackage[babel=true]{csquotes}
\usepackage{subfig}
\usepackage{wrapfig}
\usepackage{geometry}
\usepackage{listings}
\usepackage[usenames,dvipsnames,svgnames,table]{xcolor}

\geometry{top=2.7cm,bottom=2.7cm,left=2.7cm,right=2.7cm}
\interfootnotelinepenalty=10000
\widowpenalty=1000
\clubpenalty=1000
\makeatletter
\sloppy
\renewcommand\paragraph{\@startsection{paragraph}{4}{\z@}%
  {-1.5ex\@plus -1ex \@minus -.2ex}%
  {1.5ex \@plus .2ex}%
  {\normalfont\normalsize\bfseries}}
\makeatother

\newcommand{\reporttitle}{Déploiement IPv6 \\ \emph{Adressage, auto-configuration, routage, DHCPv6, mobilité}}     % Titre
\newcommand{\reportauthor}{Émilien \textsc{Kenler}\\ Abdoul \textsc{Khadre Diallo} \\ Clément \textsc{Mercier}\\ Arthur \textsc{Puyou}} % Auteur
\newcommand{\reportsubject}{Déploiement IPv6} % Sujet
\newcommand{\HRule}{\rule{\linewidth}{0.5mm}}
\setlength{\parskip}{1ex} % Espace entre les paragraphes

\lstdefinelanguage{cisco}{
	basicstyle=\small\ttfamily\color{Gray},
	moredelim=*[l][\color{black}]{\#},
	morecomment=[l][\color{Gray}]{!},
	moredelim=**[s][\color{YellowOrange}\ttfamily]{<}{>}
}

\lstset{language=cisco,                        % Use MATLAB
        frame=single,                           % Single frame around code
        basicstyle=\small\ttfamily,             % Use small true type font
        keywordstyle=[1]\color{Blue}\bf,        % MATLAB functions bold and blue
        keywordstyle=[2]\color{Purple},         % MATLAB function arguments purple
        keywordstyle=[3]\color{Blue}\underbar,  % User functions underlined and blue
        identifierstyle=,                       % Nothing special about identifiers
        commentstyle=\usefont{T1}{pcr}{m}{sl}\color{Green}\small,
        stringstyle=\color{Purple},             % Strings are purple
        showstringspaces=false,                 % Don't put marks in string spaces
        tabsize=5,                              % 5 spaces per tab
        morecomment=[l][\color{Blue}]{...},     % Line continuation (...) like blue comment
        numbers=left,                           % Line numbers on left
        firstnumber=1,                          % Line numbers start with line 1
        numberstyle=\tiny\color{Blue},          % Line numbers are blue
        stepnumber=5,                            % Line numbers go in steps of 5
        extendedchars=false
        }

\usepackage{pifont,mdframed}

\newenvironment{warning}
  {\par
    \begin{list}{}{\leftmargin=1cm
                   \labelwidth=\leftmargin}\item[\Large\ding{43}]}
  {\end{list}\par}

% Includes a figure
% The first parameter is the label, which is also the name of the figure
%   with or without the extension (e.g., .eps, .fig, .png, .gif, etc.)
% The second parameter is the width of the figure normalized to column width
%   (e.g. 0.5 for half a column, 0.75 for 75% of the column)
% The third parameter is the caption.
\newcommand{\scalefig}[3]{
  \begin{figure}[ht!]
    % Requires \usepackage{graphicx}
    \centering
    \includegraphics[width=#2\columnwidth]{#1}
    \caption{#3}
    \label{#1}
  \end{figure}}
\newcommand{\scalefiguntitled}[2]{
  \begin{figure}[ht!]
    \centering
    \includegraphics[width=#2\columnwidth]{#1}
    \label{#1}
  \end{figure}}


\hypersetup{
    pdftitle={\reporttitle},%
    pdfauthor={\reportauthor},%
    pdfsubject={\reportsubject},%
    pdfkeywords={rapport} {vos} {mots} {clés}
}

\begin{document}
  \begin{center}


\includegraphics [width=50mm]{logo-utc.pdf} \\[3cm]

\HRule \\[0.2cm]
\begin{spacing}{1.4}
{\huge \bfseries \reporttitle}\\[0.2cm]
\end{spacing}
\HRule \\[0.5cm]
\textsc{\Large SR04 -- A13}\\[2.0cm]

\begin{minipage}[t]{0.3\textwidth}
  \begin{flushleft} \large
    \emph{Étudiants} \\
    \reportauthor
  \end{flushleft}
\end{minipage}
\begin{minipage}[t]{0.6\textwidth}
  \begin{flushright} \large
    \emph{Responsable de l'UV SR04} \\
    M.~Abdelmadjid \textsc{Bouabdallah} \\
  \end{flushright}
\end{minipage}

\vfill

\end{center}
  \cleardoublepage % Dans le cas du recto verso, ajoute une page blanche si besoin
  
  \tableofcontents
  
  \cleardoublepage
    \section{Introduction}
  IPv6 est la nouvelle version du protocole IP (couche 3 du modèle OSI), succédant à IPv4, qui est aujourd'hui couramment utilisée. Elle a été notamment développée car on se rapprochait de plus en plus d’une pénurie d’adresses au niveau mondial.
  
  Cette étude présente les caractéristiques de cette nouvelle version puis quelques détails sur son implémentation.
  

  
  \cleardoublepage  
  \section{Présentation des caractéristiques d’IPv6}
\subsection{Notation des adresses IPv6}
  
\begin{itemize}
  \item La première grande différence est la longueur des adresses qui passe de 32 bits à 128 bits. Le nombre d’adresses augmente donc considérablement.
  \item On passe d’une écriture décimale (\texttt{172.69.31.15}) à une héxadécimale (\texttt{2001:0db8:0000:85a3:0000:0000:ac1f:8001})
  \item Il est permis de ne pas noter les zéros non significatifs. Ainsi l'adresse précédente devient : \texttt{2001:db8:0:85a3:0:0:ac1f:8001}
  \item Des groupes de bits nuls peuvent être omis en gardant les deux-points. Par exemple, dans \texttt{2001:db8:0:85a3::ac1f:8001}, on sait qu'il s’agit de groupes nuls à la place de \texttt{::}. Cette substitution ne peut se faire qu’à un seul endroit, sinon l’adresse devient ambiguë : \texttt{2001:db8::85a3::ac1f:8001} n'est pas valide car on ne sait pas où sont les groupes nuls et comment repartir leur nombre.
  \item Il n’y a plus de masques sous la forme de quatre groupes d'un octet, comme avec IPv4.
Seule la notation CIDR a été conservée.
Elle consiste en un slash (\texttt{/}), suivi de la longueur du préfixe réseau.
Le couple formé par le préfixe réseau et la longueur du préfixe permet de savoir de quel type d’adresse il s’agit.
\end{itemize}

\subsection{Types d’adresse IPv6}

\begin{itemize}
  \item	\texttt{::/128} : adresse non spécifiée, c’est le \emph{this} pour l’attribution d’une adresse.
  \item	\texttt{::1/128} : adresse loopback, la machine elle-même.
  \item	\texttt{fc00::/7} : adresse locale unique, équivalent des adresses privées.
  \item \texttt{fe80::/10} : adresse locale de lien pour communiquer avec les appareils à distance \texttt{1}, c’est-à-dire les équipements reliés directement.
On dit que l’adresse est restreinte à un lien.
Nous avons déjà le préfixe : les 64 derniers bits dédiés à l’adresse hôte ont la valeur de l’identifiant d’interface, basée sur l’adresse MAC de la machine.
  \item \texttt{ff00::/8} : adresses multicast.
  \item \texttt{ff02::2} : adresse multicast spécifique permettant de s’adresser à tous les routeurs de distance \texttt{1}.
  \item \texttt{ff02::1} : adresse multicast spécifique permettant de s’adresser à tous les hôtes voisins de distance \texttt{1}.
  \item \texttt{/64} : préfixe d’un sous réseau. La taille d’une sous réseau est fixe en IPv6.
Il y a 48 bits pour le FAI, 16 bits pour le sous réseau et 64 bits pour l’hôte.
\end{itemize}

\subsection{Principe des options}

La plupart des en-têtes des paquets sont désormais optionnelles, elle n'ont plus besoin d'être présentes dans l'en-tête.
Cela permet d’alléger de manière générale la taille d’un paquet tout en permettant une plus grande flexibilité et évolutivité.
Les options de l’en-tête sont stockées sous la forme d’une liste chaînée.

\newpage
\section{Attribution des adresses en IPv6}

\subsection{Auto-configuration}

Il existe différentes méthodes avec IPv6 pour l’attribution d’une adresse IP globale aux équipements.

\begin{itemize}
  \item La classique attribution manuelle.
  \item L’auto-configuration sans état, en lien avec le NDP.
Il n’y a pas besoin de serveur pour distribuer les adresses, cette dernière est auto-attribuée grâce à NDP.
On a ici un fonctionnement \emph{plug-and-play} au niveau de l’interface réseau.
  \item L’auto-configuration avec état, c’est-à-dire avec un serveur DHCPv6, très semblable à celle sous IPv4.
On garde la main sur les adresses attribuées en donnant un pool d’adresse au serveur DHCP, permettant de centraliser les adresses attribuées.
Nous ne détaillerons pas plus cela. % TODO retirer après avoir fait la partie DHCPv6
\end{itemize}

\subsection{Neighbor Discovery Protocol (NDP)}

Nous allons ici faire le lien entre l’adresse locale de lien et l’auto-configuration sans état.
Le NDP est une des nouveautés d’IPv6 et c’est ce qui contribue à son attrait.
Ce protocole fournit des services semblables à ARP (Address Resolution Protocol), ICMP (Internet Control and Error Messages) et plus encore.
Il utilise principalement le protocole ICMP qui a été enrichi par rapport à IPv4 : on parle d’ICMPv6.

Voici les différentes fonctions que remplit ce protocole :
\begin{itemize}
  \item La résolution d’adresse, semblable à ARP seulement ici le protocole utilise les messages standards d’ICMPv6.
  \item NUD (Neighbor Unreachability Detection) : c’est une des nouveautés d’IPv6, cela permet d’effacer des tables les équipements proches devenus inaccessibles. Il permet une plus grande adaptabilité pour changer une route au sein d’un routeur si un de ses voisins devient inaccessible.
  \item La configuration automatique d’un équipement grâce à la découverte de ses voisins avec par exemple :
    \begin{itemize}
      \item découverte des routeurs;
      \item découverte des préfixes réseau, qui lui permet de construire son ou ses propres adresses IPv6;
      \item découverte des adresses dupliquées, qui permet d’éviter que l’adresse construite automatiquement n’existe déjà sur le réseau;
      \item découverte des paramètres du lien physique (taille du MTU, nombre de sauts max autorisés...).
    \end{itemize}
  \item	Indication de redirection, envoyée par un routeur s’il connaît une meilleure route en nombre de saut pour atteindre une destination.
\end{itemize}On peut donc voir que IPv6 remplit de nombreuses fonctions dont certaines sont nouvelles.

\newpage
Voyons donc la procédure d’auto-configuration d’une adresse pour un nouvel équipement rattaché à un réseau :
\begin{itemize}
  \item la toute première étape consiste à créer l'adresse locale de lien (\textit{link-local address}) de l’équipement.
  \item une fois l'unicité de cette adresse vérifiée, la machine est en mesure de communiquer avec les autres machines de distance 1. Les machines de distance 1 sont les machines du lien.
  \item pour savoir comment sera attribuée l’adresse unicast globale (l’IP globale), la machine doit chercher à acquérir un message d'annonce (\textit{Router Advertisement}) d’un routeur.
  \item s'il y a un routeur sur le lien, la machine doit appliquer la méthode indiquée par le message d'annonce du routeur :
    \begin{itemize}
      \item l'auto-configuration sans état, la machine utilise sont adresse de lien locale comme adresse unicast globale;
      \item l'auto-configuration avec état (DHCPv6).
    \end{itemize}
  \item en l'absence de routeur sur le lien, la machine doit essayer d'acquérir l'adresse unicast globale par la méthode d'auto-configuration sans état. Si la tentative échoue, c'est terminé. Les communications se feront uniquement sur le lien avec l'adresse locale de lien. La machine n'a pas une adresse avec une portée qui l'autorise à communiquer avec des machines autres que celles du lien.
\end{itemize}


  
  \cleardoublepage
  \section{Mobilité IPv6 (MIPv6)}


Pour communiquer avec des serveurs, des connexions sont établies et elles sont liées à l’adresse IP.
Le problème lié à la mobilité est que l’adresse IP est liée au réseau sur lequel on se trouve, dès lors que l’on va se déplacer on va changer de réseau et obtenir une nouvelle adresse IP, les connexions actives vont donc être perdues car l’adresse IP avec laquelle elles avaient établi les connexions n’est plus joignable.
Un mécanisme à donc été mis en place au sein d’IPv6 pour résoudre le problème : il faut autoriser un équipement à garder son adresse IP \textbf{même s’il change de réseau}.
Pour cela un équipement est attaché à un réseau, ce sera son \textbf{réseau mère}, qui lui fournira une adresse IP fixe et ce dernier permettra d’assurer la continuité des connexions établies.
On parle de \emph{home address} (HoA) pour l’adresse IP fixe liée au réseau mère (\emph{home network}).
De plus l’équipement aura toujours une adresse IP variable et temporaire liée à sa localisation et au réseau auquel il est directement connecté, on parle de \emph{care-of address} (CoA).
La pierre angulaire de ce système est l’agent mère (HA) situé sur le réseau mère et chargé d’assurer la correspondance entre la HoA et la CoA.

L’IPv6 étant un protocole construit en amont par l’IETF, certains mécanismes pour la mobilité ont directement pu être intégrés dans la conception du protocole.

Voici un petit schéma pour résumer et expliciter tout cela :

\scalefig{images/schema1}{.7}{}

On retrouve le nœud correspondant, qui est le serveur à joindre, le réseau mère qui fait la liaison entre les 2 adresses et le réseau étranger où se trouve physiquement notre équipement.
La HoA se situe donc sur le réseau mère et la CoA se situe sur le réseau étranger.
Voyons maintenant les cas d’utilisations de ce système.

\subsection{Le noeud mobile est dans son réseau mère}

C’est le cas le plus simple, il utilise directement sa \emph{Home Address} et on a un routage classique pour la communication entre le noeud mobile et le serveur.

\scalefig{images/schema2}{.7}{}

\subsection{Le mobile est dans un réseau étranger}

C’est là que cela devient intéressant, le mobile aura donc sa HoA et une adresse étrangère CoA acquise par les mécanismes d’auto-configuration sur le réseau étranger où est physiquement le noeud mobile.
L’agent mère va donc tenir sa table d’association et associer la HoA avec la CoA.
Si un correspondant veut envoyer un paquet au noeud mobile il enverra un paquet avec comme source la sienne et comme destination la \emph{Home Address} du noeud mobile.
Une fois arrivé à l’agent mère le paquet va être routé vers la CoA du noeud mobile.
Pour cela l’agent mère encapsule le paquet qui a donc comme source l’adresse de l’agent mère et comme destination la CoA du mobile.
Une fois arrivé au mobile le paquet est désencapsulé et transmis aux couches supérieurs avec donc comme source et destinations celles du premier paquet.
Au yeux du mobile c’est comme s’il était sur son réseau mère, les connexions ne sont donc plus coupées car même si la CoA change, au yeux du mobile la destination est toujours sa HoA qui ne change pas.
L’encapsulation se fait grâce à l’extension d’en-tête IP-IP d’IPv6, dès lors les paquets sont protégés par IPsec et tunnelé vers la CoA en toute sécurité.

\scalefig{images/schema3}{.7}{}
\scalefig{images/schema4}{.7}{}

\subsection{Mécanisme d’association}

Pour créer ou mettre à jour une association au niveau de l’agent mère, lors d’un changement de réseau du noeud mobile par exemple, un mécanisme spécifique de Mobilité IPv6 est utilisé.
Il s’agit du \emph{Binding Update (BU)}, appelé en français mise à jour d’association.
Ce mécanisme utilise un mécanisme du NDP permettant de savoir qu’un hôte n’est plus accessible.

\scalefig{images/schema5}{.7}{}

Un soucis que pose ce mécanisme est une certaines inefficacité et une lenteur au niveau du routage car il faut obligatoirement passé par le réseau mère, même si le réseau avec lequel on communique est proche de nous.
Un mécanisme d’optimisation du routage a donc été mis en place, mais il n’est pas toujours supporté par le serveur avec lequel on communique.
Il faut donc garder à l’esprit que la mobilité se fait bien sans cela, il ne s’agit que d’une amélioration, pas d’une obligation.

\subsection{Optimisation du routage}

Pour gagner en efficacité dans certains cas, l’optimisation de routage a été mise en place.
Le principe est simple et reprend le mécanisme général avec le réseau mère, seulement ici le noeud correspondant va aussi tenir une table d’association HoA/CoA, comme l'agent mère.
Ainsi par ce biais-là, on garde le même mécanisme, mais le noeud correspondant se substitue à l'agent mère.
Le noeud mobile contacte directement au noeud correspondant, qui va enlever l’en-tête et obtenir un paquet venant de la HoA et à destination du CN, la connexion n’est donc pas brisée.

\scalefig{images/schema6}{.7}{}
\scalefig{images/schema7}{.7}{}

Il faut aussi faire des Binding Update pour garder à jour les tables d’associations.
Le noeud mobile doit donc connaître les agents auxquels il doit envoyer ces mises à jour.
Il a donc une table des agents et lors d’un BU il enverra à son agent mère et au noeud correspondant supportant l’optimisation de routage.
Il doit toujours tenir à jour l’agent mèrei, car certaines connexions ne supportent pas l’optimisation suivant le serveur correspondant, il peut donc y avoir à la fois des connexions optimisées et d’autres non optimisées.

\scalefig{images/schema8}{.7}{}


  
  \cleardoublepage
  \section{Mise en œuvre d'IPv6}

Afin d'illustrer ce projet par une démonstration technique, un routeur \emph{Cisco 1811} nous a été prêté. L'objectif est de créer deux réseaux et de faire basculer un client d'un réseau à un autre, tout en restant joignable sur son adresse IPv6 d'origine.

\subsection{Installation}

Afin de permettre à tous de travailler sur le routeur, il a été installé de manière fixe chez Émilien et relié à une \emph{Freebox v6}.
Cette Freebox dispose d'une connexion IPv6 par défaut et chaque abonné dispose de 8 sous-réseaux \texttt{/64}.

Les branchements sont les suivants :
\begin{center}
    \begin{tabular}{|l|l|}
    \hline
    Port                   & Connecté à \\ \hline
    FastEthernet 0 (WAN)   & Freebox    \\ \hline
    FastEthernet 1 (WAN)   & n.c.       \\ \hline
    FastEthernet 2-8 (LAN) & Clients    \\ \hline
    \end{tabular}
\end{center}

On commence la configuration du routeur en se connectant sur le port console. On commence donc par lui attribuer un nom et un domaine :

\begin{lstlisting}
router(config)#hostname router
router(config)#ip domain-name emilienkenler.net
\end{lstlisting}

On peut ensuite activer le SSH :

\begin{lstlisting}
router(config)#ip ssh rsa keypair-name sshkey
router(config)#crypto key generate rsa usage-keys label sshkeys \
               modulus 2048
router(config)#ip ssh version 1
router(config)#ip ssh logging events
router(config)#ip ssh time-out 60
router(config)#ip ssh authentication-retries 5
router(config)#service password-encryption
router(config)#enable secret sr04
router(config)#username sr04 password 0 sr04
router(config)#line vty 0 4
router(config-line)#login local
router(config-line)#transport input ssh
\end{lstlisting}

On définit ensuite la Freebox comme route par défaut en IPv4 :
\begin{lstlisting}
router(config)#ip route 0.0.0.0 0.0.0.0 192.168.1.254
\end{lstlisting}

On active enfin l’IPv6 sur l’interface raccordée à la Freebox ainsi que l’auto-configuration, dans le but de récupérer une adresse IPv6 dans le réseau géré par la Freebox :

\begin{lstlisting}
router(config)#ipv6 unicast-routing 
router(config)#int FastEthernet 0
router(config-if)#ipv6 enable
router(config-if)#ipv6 address autoconfig default
router#show ipv6 route
S   ::/0 [2/0]
     via FE80::224:D4FF:FE9A:18E0, FastEthernet0
\end{lstlisting}

On peut ensuite récupérer l'adresse IPv6 utilisée par le routeur :
\begin{lstlisting}
router#sh ipv6 interface FastEthernet 0
  IPv6 is enabled, link-local address is FE80::5A8D:9FF:FE14:8900 
  No Virtual link-local address(es):
  Stateless address autoconfig enabled
  Global unicast address(es):
    2A01:E35:8A38:6870:5A8D:9FF:FE14:8900, subnet is 2A01:E35:8A38:6870::/64 [EUI/CAL/PRE]
      valid lifetime 86004 preferred lifetime 86004
    2A01:E35:8A38:6875:5A8D:9FF:FE14:8900, subnet is 2A01:E35:8A38:6875::/64 [EUI]
  Joined group address(es):
    FF02::1
    FF02::2
    FF02::1:FF14:8900
  MTU is 1500 bytes
  ICMP error messages limited to one every 100 milliseconds
  ICMP redirects are enabled
  ICMP unreachables are sent
  ND DAD is enabled, number of DAD attempts: 1
  ND reachable time is 30000 milliseconds (using 18502)
  ND advertised reachable time is 0 (unspecified)
  ND advertised retransmit interval is 0 (unspecified)
  ND router advertisements are sent every 200 seconds
  ND router advertisements live for 1800 seconds
  ND advertised default router preference is Medium
  Hosts use stateless autoconfig for addresses.
\end{lstlisting}

L'adresse du routeur est donc \texttt{2a01:e35:8a38:6870:5a8d:9ff:fe14:8900}.
Pour éviter de devoir taper cette adresse à chaque connexion, on créé une entrée DNS\footnote{\emph{Domain Name System}} de type \texttt{AAAA} avec le nom \texttt{router.emilienkenler.net} :

\begin{lstlisting}
router.emilienkenler.net. 1651  IN      AAAA    2a01:e35:8a38:6870:5a8d:9ff:fe14:8900
\end{lstlisting}

Au sein de l'interface de la \emph{Freebox v6}, il est possible de demander le routage de certains réseau vers une adresse qui se trouve derrière la \emph{Freebox}.
Le routeur est alors appelé le \emph{Next Hop} : c'est le point de passage suivant pour tous les paquets adressés à ce sous-réseau.
Ceci va nous permettre d'accéder à chacun des réseaux que nous utilisons pour la démonstration depuis n'importe quel endroit sur Internet.

Dans notre cas, le réseau par défaut de la \emph{Freebox} est \texttt{2a01:e35:8a38:6870::/64}.
Dans l'interface de la Freebox, on accède donc au menu \emph{Paramètres de la freebox} > \emph{Mode avancé} > \emph{Configuration IPv6}, puis on configure les 4 sous-réseaux qui seront utilisés par le projet pour les faire pointer vers le routeur :

\begin{center}
    \begin{tabular}{|l|l|l|l|}
    \hline
    ID du sous-réseau & Adresse du sous-réseau           & Description                          & Interface \\ \hline
    2                 & \texttt{2a01:e35:8a38:6872::/64} & Réseau domicile                      & VLAN2     \\ \hline
    3                 & \texttt{2a01:e35:8a38:6873::/64} & Réseau mobile                        & VLAN3     \\ \hline
    4                 & \texttt{2a01:e35:8a38:6874::/64} & \emph{Réservé Wi-Fi}                       & -         \\ \hline
    5                 & \texttt{2a01:e35:8a38:6875::/64} & Réseau extérieur & WAN       \\ \hline
    \end{tabular}
\end{center}

% screenshot de la freebox

La configuration initiale est maintenant terminée et on peut débrancher la console.

Nous n'avons pas réussi à nous connecter avec la version 2 du protocole SSH.
La version est donc forcée à 1 dans la configuration du routeur et il faut se connecter avec la commande suivante :

\begin{lstlisting}
$ ssh -6 -1 sr04@router.emilienkenler.net
\end{lstlisting}

\subsection{Auto-configuration IPv6}

On utilise le sous-réseau \texttt{2a01:e35:8a38:6875::/64} pour tester la mise en place de l'auto-configuration au niveau du routeur.
Ce réseau est configuré sur l'interface WAN du routeur.
Comme ce sous-réseau a été délégué au routeur dans la configuration de la Freebox, l'auto-configuration est désactivée de son côté.
En activant l'auto-configuration du côté du routeur, tout périphérique connecté du côté WAN (par l'intermédiaire du switch de la Freebox), pourra négocier une adresse IPv6 avec le routeur.

On active la configuration comme suit :

\begin{lstlisting}
router(config)#int FastEthernet 0
router(config-if)#ipv6 enable
router(config-if)#ipv6 address 2a01:e35:8a38:6875::/64 eui-64
router(config-if)#ipv6 nd prefix 2a01:e35:8a38:6875::/64
\end{lstlisting}

Pour vérifier que cela fonctionne, on peut déjà vérifier que le routeur a obtenu automatiquement une adresse sur son interface WAN :
\begin{lstlisting}
router#sh ipv6 interface FastEthernet 0
FastEthernet0 is up, line protocol is up
  IPv6 is enabled, link-local address is FE80::5A8D:9FF:FE14:8900 
  No Virtual link-local address(es):
  Global unicast address(es):
    2A01:E35:8A38:6870:5A8D:9FF:FE14:8900, subnet is 2A01:E35:8A38:6870::/64
    2A01:E35:8A38:6875:5A8D:9FF:FE14:8900, subnet is 2A01:E35:8A38:6875::/64 [EUI]
\end{lstlisting}

\subsection{Configuration des VLAN}

Afin de séparer les réseaux domicile et mobile entre les différents ports du switch, on créé différents VLAN.

Le VLAN2 est utilisé pour le réseau domicile.
On le rend accessible sur les ports 2 et 3 du switch :

\begin{lstlisting}
router(config)#vlan 2
router(config-vlan)#name VLAN2

router(config)#interface FastEthernet 2
router(config-if)#switchport mode access
router(config-if)#switchport access vlan 2

router(config)#interface fastEthernet 3
router(config-if)#switchport mode access
router(config-if)#switchport access vlan 2
\end{lstlisting}

On indique ensuite l'adresse du sous-réseau au routeur et on active l'auto-configuration :

\begin{lstlisting}
router(config)#interface vlan 2
router(config-if)#ipv6 enable                              
router(config-if)#ipv6 address 2a01:e35:8a38:6872::/64 eui-64
router(config-if)#ipv6 nd ra interval 20
router(config-if)#ipv6 nd advertisement-interval
router(config-if)#ipv6 nd prefix 2a01:e35:8a38:6872::/64
\end{lstlisting}

La même configuration est appliquée pour le VLAN3, qui est utilisé pour le réseau mobile sur les ports 4 et 5 :

\begin{lstlisting}
router(config)#vlan 3
router(config-vlan)#name VLAN3

router(config)#interface FastEthernet 4
router(config-if)#switchport mode access
router(config-if)#switchport access vlan 3

router(config)#interface FastEthernet 5
router(config-if)#switchport mode access
router(config-if)#switchport access vlan 3

router(config)#interface vlan 3
router(config-if)#ipv6 enable       
router(config-if)#ipv6 address 2a01:e35:8a38:6873::/64 eui-64
router(config-if)#ipv6 nd ra interval 20
router(config-if)#ipv6 nd advertisement-interval
router(config-if)#ipv6 nd prefix 2a01:e35:8a38:6873::/64
\end{lstlisting}

\subsection{Configuration du Wi-Fi}

Nous souhaitions initialement réaliser une démonstration consistant à passer un client d'une connexion filaire à une connexion Wi-Fi.
La conservation de la même IP devrait alors permettre de faire cette bascule sans interruption des connexions.

Cependant, l'interface Wi-Fi du routeur dont nous disposons fonctionne en mode bridge avec une autre interface et les bridge ne supportent pas l'IPv6.
Il n'est donc pas possible de faire de l'IPv6 en Wi-Fi avec ce routeur.

La configuration du Wi-Fi été abandonnée.
La configuration du Wi-Fi en IPv4 est disponible en annexe à titre de référence.

\subsection{Configuration de Mobile IPv6 sur le routeur}

Afin que les clients puissent utiliser Mobile IPv6 sur les réseaux VLAN2 et VLAN3, on configure le routeur comme un \emph{Home Agent} :

\begin{lstlisting}
router(config)#interface vlan 2
router(config-if)#ipv6 mobile home-agent
router(config)#interface vlan 3
router(config-if)#ipv6 mobile home-agent
\end{lstlisting}

On peut alors afficher des détails sur l'état de cette fonctionnalité :

\begin{itemize}

\item les informations générales :
\begin{lstlisting}
router#sh ipv6 mobile globals
Mobile IPv6 Global Settings:

  3 Home Agent service on following interfaces:
    FastEthernet0
    Vlan2
    Vlan3
  Features:
    Auth-option support disabled
  Bindings:
    Maximum number is unlimited.
    0 bindings are in use
    1 bindings peak
    Binding lifetime permitted is 262140 seconds
    Recommended refresh time is 300 
\end{lstlisting}

\item les détails relatifs à chaque interface :
\begin{lstlisting}
router#sh ipv6 mobile home-agent
Home Agent information for Vlan2
  Configured:
    FE80::5A8D:9FF:FE14:8900
    preference 0 lifetime 1800
      global address 2A01:E35:8A38:6872:5A8D:9FF:FE14:8900/64

  No Discovered Home Agents
Home Agent information for Vlan3
  Configured:
    FE80::5A8D:9FF:FE14:8900
    preference 0 lifetime 1800
      global address 2A01:E35:8A38:6873:5A8D:9FF:FE14:8900/64

  No Discovered Home Agents
\end{lstlisting}

\item la liste des machines connectées
\begin{lstlisting}
router#sh ipv6 mobile binding 
Mobile IPv6 Binding Cache Entries:

Selection matched 0 bindings
\end{lstlisting}

\item des détails sur le traffic
\begin{lstlisting}
router#sh ipv6 mobile traffic
MIPv6 statistics:
  Rcvd:  111 total
      0 truncated, 0 format errors
      0 checksum errors
    Binding Updates received: 109
      0 no HA option, 0 BU's length
      0 options' length, 0 invalid CoA
  Sent:  174 generated
    Binding Acknowledgements sent: 109
      107 accepted (0 prefix discovery required)
      0 reason unspecified, 0 admin prohibited
      0 insufficient resources, 0 home reg not supported
      0 not home subnet, 2 not home agent for node
      0 DAD failed, 0 sequence number
    Binding Errors sent: 65
      63 no binding, 2 unknown MH

  Home Agent Traffic:
    98 registrations, 9 deregistrations
    2w1d since last accepted HA registration
    2w1d since last failed HA registration
    133 last failed registration code
    Traffic forwarded:
      0 tunneled, 0 reversed tunneled
    Dynamic Home Agent Address Discovery:
      0 requests received, 0 replies sent
    Mobile Prefix Discovery:
      2036 solicitations received, 2036 advertisements sent\end{lstlisting}
\end{itemize}

En cas de besoin, on peut activer l'affichage de messages supplémentaires dans les journaux relatifs à cette fonctionnalité :

\begin{lstlisting}
router#debug ipv6 mobile AR-selection
router#debug ipv6 mobile binding-cache
router#debug ipv6 mobile forwarding 
router#debug ipv6 mobile home-agent   
router#debug ipv6 mobile networks    
router#debug ipv6 mobile registrations      
router#debug ipv6 mobile router 
\end{lstlisting}

\subsection{Configuration du client}

On souhaite configurer le client en tant que \emph{Mobile Node}.

Il existe plusieurs implémentations d'IPv6 sous Linux.
Nous avons choisi \emph{UMIP} \footnote{Universal Mobile IP for Linux, \url{http://www.umip.org/}}, qui est une implementation de la RFC6275\footnote{\url{http://tools.ietf.org/html/rfc6275}}.
Une autre alternative est \emph{MIPL} \footnote{Mobile IPv6 for Linux}, dont une documentation est encore disponible \footnote{\url{http://tldp.org/HOWTO/Mobile-IPv6-HOWTO/mipv6.html}}, mais dont le site n'est plus accessible.

UMIP nécessite le support du noyau pour pouvoir fonctionner. Sous Fedora, ce support est disponible nativement et le paquet \texttt{mipv6-daemon} est disponible pour l'exploiter.

\begin{lstlisting}
# yum install mipv6-daemon
\end{lstlisting}

On peut alors créer le fichier de configuration \texttt{/etc/mip6d.conf}, qui spécifie que le démon doit agir en tant que \emph{Mobile Node} (MN), ainsi que l'adresse du réseau domicile et du \emph{Home Agent} (c'est-à-dire le routeur) :

\begin{lstlisting}
NodeConfig MN;
DebugLevel 10;
OptimisticHandoff enabled;
DoRouteOptimizationMN disabled;
MnMaxHaBindingLife 60;

Interface "em1" {
  MnIfPreference 1;
}

MnHomeLink "em1" {
  HomeAgentAddress 2A01:E35:8A38:6872:5A8D:9FF:FE14:8900;
  HomeAddress 2A01:E35:8A38:6872::1/64;
}
\end{lstlisting}

On peut ensuite redémarrer le démon pour prendre en compte cette configuration :

\begin{lstlisting}
# systemctl restart mip6d.service
\end{lstlisting}

\subsection{Analyse du traffic}

Maintenant que tout est configuré, on peut réaliser la démonstration.
L'objectif est de connecter un client dans le réseau domicile, puis de le connecter dans le réseau mobile et de vérifier qu'il est toujours joignable sur son adresse d'origine.
Pour vérifier cela, on utilisera un tiers qui envoie des \texttt{ping} vers l'adresse dans le réseau domicile.
Le tiers pourra se trouver dans un réseau quelconque.

\begin{warning}
Une première remarque importante à faire concerne la gestion de l'état des interfaces VLAN par le routeur.

Conformément à une note sur le site de Cisco\footnote{\url{https://supportforums.cisco.com/docs/DOC-2032}}, une interface VLAN n'est disponible que si au moins un périphérique est connecté à une interface dans ce VLAN.

Ainsi, si le client mobile de test est le seul périphérique connecté au switch et qu'on le branche sur un port du VLAN mobile, les ports 2 et 3 ne sont pas connectés.
Le switch va alors désactiver le VLAN2 et passer l'interface à l'état \texttt{up/down} au lieu de \texttt{up/up}.

Si on regarde l'état du Mobile IPv6 sur le routeur dans ces conditions, on constate effectivement qu'aucune adresse n'est détectée sur l'interface VLAN2 :
\begin{lstlisting}
router#sh ipv6 mobile home-agent
Home Agent information for Vlan2
  Configured:
    FE80::5A8D:9FF:FE14:8900
    preference 0 lifetime 1800
      No global addresses within prefix

  No Discovered Home Agents
Home Agent information for Vlan3
  Configured:
    FE80::5A8D:9FF:FE14:8900
    preference 0 lifetime 1800
      global address 2A01:E35:8A38:6873:5A8D:9FF:FE14:8900/64

  No Discovered Home Agents
\end{lstlisting}

Une alternative consiste à utiliser le VLAN dans lequel se trouve la Freebox, mais celle-ci va alors répondre aux trames liées à Mobile IPv6, ce qui perturbe le client.

Il faut donc veiller à laisser un périphérique connecté en permanence dans le réseau domicile.
\end{warning}

% TODO Descriptif de la capture

\subsection{Sécurisation}


  \cleardoublepage
  \section{Conclusion}

Cette étude nous a permis d'acquérir de bonnes connaissances de base sur IPv6, qui va devenir très rapidement omniprésent.
Mobile IPv6 est une fonctionnalité supplémentaire apportée par IPv6.
Elle montre que le protocole permet une construction modulaire : la sécurité de Mobile IPv6 peut ainsi être implémentée de différentes manières, tout en conservant un protocole de base compatible.

Cependant, cette modularité est aussi une faiblesse : Cisco et UMIP n'ont pas choisi d'implémenter les mêmes techniques pour la sécurisation, ce qui pose des soucis d'interopérabilité.
Cette modularité fait penser à l'authentification sur les réseaux Wi-Fi, que nous avons étudiée en SR06.
La RFC5778\footnote{\url{http://tools.ietf.org/html/rfc5778}} décrit d'ailleurs une méthode pour s'authentifier avec de l'EAP en Mobile IPv6, un protocole qui est couramment utilisé sur les réseaux Wi-Fi aujourd'hui.

Enfin, 10 ans après la publication de la première RFC sur Mobile IPv6, son support reste relativement limité : une des implémentations sous Linux a été abandonnée et l'utilisation sous les dernières variantes de Debian nécessite une compilation de noyau.
De plus, le processus nécessite une configuration manuelle qui réserve pour l'instant son utilisation à des cas très précis.

Mobile IPv6 nécessite donc encore quelques développements avant d'être utilisable à grande échelle, mais apporte une fonctionnalité précieuse à IPv6.
Le nécessaires sera donc très probablement réalisé avec le déploiement d'IPv6 dans les années qui viennent.


%                                                               %
%                        \                                      %
%                         \                                     %
%                          \\                                   %
%                           \\                                  %
%                            >\/7                               %
%                        _.-(6'  \                              %
%                       (=___._/` \                             %
%                            )  \ |                             %
%                           /   / |                             %
%                          /    > /                             %
%                         j    < _\                             %
%                     _.-' :      ``.                           %
%                     \ r=._\        `.                         %
%                    <`\\_  \         .`-.                      %
%                     \ r-7  `-. ._  ' .  `\                    %
%                      \`,      `-.`7  7)   )                   %
%                       \/         \|  \'  / `-._               %
%                                  ||    .'                     %
%                                   \\  (                       %
%                                    >\  >                      %
%                                ,.-' >.'                       %
%                               <.'_.''                         %
%                                 <'                            %
%                                                               %

  \cleardoublepage
  \listoffigures
\end{document}
