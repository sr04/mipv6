\documentclass[a4paper,11pt,final]{article}
% Pour une impression recto verso, utilisez plutôt ce documentclass :
%\documentclass[a4paper,11pt,twoside,final]{article}

\usepackage[english,francais]{babel}
\usepackage[utf8]{inputenc}
\usepackage[T1]{fontenc}
\usepackage[pdftex]{graphicx}
\usepackage{setspace}
\usepackage{hyperref}
\usepackage[french]{varioref}
\usepackage[babel=true]{csquotes}
\usepackage{subfig}
\usepackage{wrapfig}
\usepackage{geometry}
\geometry{top=2.7cm,bottom=2.7cm,left=2.7cm,right=2.7cm}
\interfootnotelinepenalty=10000
\widowpenalty=1000
\clubpenalty=1000
\makeatletter
\renewcommand\paragraph{\@startsection{paragraph}{4}{\z@}%
  {-1.5ex\@plus -1ex \@minus -.2ex}%
  {1.5ex \@plus .2ex}%
  {\normalfont\normalsize\bfseries}}
\makeatother

\newcommand{\reporttitle}{Déploiement IPv6 \\ \emph{Adressage, auto-configuration, routage, DHCPv6, mobilité}}     % Titre
\newcommand{\reportauthor}{Émilien \textsc{Kenler}\\ Abdoul \textsc{Khadre Diallo} \\ Clément \textsc{Mercier}\\ Arthur \textsc{Puyou}} % Auteur
\newcommand{\reportsubject}{Déploiement IPv6} % Sujet
\newcommand{\HRule}{\rule{\linewidth}{0.5mm}}
\setlength{\parskip}{1ex} % Espace entre les paragraphes

% Includes a figure
% The first parameter is the label, which is also the name of the figure
%   with or without the extension (e.g., .eps, .fig, .png, .gif, etc.)
% The second parameter is the width of the figure normalized to column width
%   (e.g. 0.5 for half a column, 0.75 for 75% of the column)
% The third parameter is the caption.
\newcommand{\scalefig}[3]{
  \begin{figure}[ht!]
    % Requires \usepackage{graphicx}
    \centering
    \includegraphics[width=#2\columnwidth]{#1}
    \caption{#3}
    \label{#1}
  \end{figure}}
\newcommand{\scalefiguntitled}[2]{
  \begin{figure}[ht!]
    \centering
    \includegraphics[width=#2\columnwidth]{#1}
    \label{#1}
  \end{figure}}


\hypersetup{
    pdftitle={\reporttitle},%
    pdfauthor={\reportauthor},%
    pdfsubject={\reportsubject},%
    pdfkeywords={rapport} {vos} {mots} {clés}
}

\begin{document}
\begin{center}


\includegraphics [width=50mm]{logo-utc.pdf} \\[3cm]

\HRule \\[0.2cm]
\begin{spacing}{1.4}
{\huge \bfseries \reporttitle}\\[0.2cm]
\end{spacing}
\HRule \\[0.5cm]
\textsc{\Large SR04 -- A13}\\[2.0cm]

\begin{minipage}[t]{0.3\textwidth}
  \begin{flushleft} \large
    \emph{Étudiants} \\
    \reportauthor
  \end{flushleft}
\end{minipage}
\begin{minipage}[t]{0.6\textwidth}
  \begin{flushright} \large
    \emph{Responsable de l'UV SR04} \\
    M.~Abdelmadjid \textsc{Bouabdallah} \\
  \end{flushright}
\end{minipage}

\vfill

\end{center}
  \cleardoublepage % Dans le cas du recto verso, ajoute une page blanche si besoin
  \tableofcontents % Table des matières
  \sloppy          % Justification moins stricte : des mots ne dépasseront pas des paragraphes
  \cleardoublepage
  \section{Introduction}
  IPv6 est la nouvelle version du protocole IP (couche 3 du modèle OSI), succédant à IPv4, qui est aujourd'hui couramment utilisée. Elle a été notamment développée car on se rapprochait de plus en plus d’une pénurie d’adresses au niveau mondial.
  
  Cette étude présente les caractéristiques de cette nouvelle version puis quelques détails sur son implémentation.
  
  \cleardoublepage  
  \section{Présentation des caractéristiques d’IPv6}
  \subsection{Notation des adresses IPv6}
  
\begin{itemize}
\item La première grande différence est la longueur des adresses qui passe de 32 bits à 128 bits. Le nombre d’adresses augmente donc considérablement.
\item On passe d’une écriture décimale (\texttt{172.69.31.15}) à une héxadécimale (\texttt{2001:0db8:0000:85a3:0000:0000:ac1f:8001})
\item Il est permis de ne pas noter les zéros non significatifs. Ainsi l'adresse précédente devient : \texttt{2001:db8:0:85a3:0:0:ac1f:8001}
\item Des groupes de bits nuls peuvent être omis en gardant les deux-points. Par exemple, dans \texttt{2001:db8:0:85a3::ac1f:8001}, on sait qu'il s’agit de groupes nuls à la place de \texttt{::}. Cette substitution ne peut se faire qu’à un seul endroit, sinon l’adresse devient ambiguë : \texttt{2001:db8::85a3::ac1f:8001} n'est pas valide car on ne sait pas où sont les groupes nuls et comment repartir leur nombre.
\item Il n’y a plus de masques, seulement la notation CIDR \texttt{/}, suivie de la longueur du préfixe réseau. Le couple préfixe réseau/longueur du préfixe permet de savoir de quel type d’adresse il s’agit.
\end{itemize}

\subsection{Types d’adresse IPv6}

\begin{itemize}
\item	\texttt{::/128} : adresse non spécifiée, c’est le \emph{this} pour l’attribution d’une adresse\item	\texttt{::1/128} : adresse loopback, la machine elle-même\item	\texttt{fc00::/7} : adresse locale unique, équivalent des adresses privées\item \texttt{fe80::/10} : adresse locale de lien pour communiquer avec les appareils à distance \texttt{1}, c’est-à-dire les équipements reliés directement. On dit que l’adresse est restreinte à un lien. Nous avons déjà le préfixe : les 64 derniers bits dédiés à l’adresse hôte ont la valeur de l’identifiant d’interface, basée sur l’adresse MAC de la machine\item \texttt{ff00::/8} : adresses multicast\item \texttt{ff02::2} : adresse multicast spécifique permettant de s’adresser à tous les routeurs de distance \texttt{1}\item \texttt{ff02::1} : adresse multicast spécifique permettant de s’adresser à tous les hôtes voisins de distance \texttt{1}\item \texttt{/64} : préfixe d’un sous réseau. La taille d’une sous réseau est fixe en IPv6. Il y a 48 bits pour le FAI, 16 bits pour le sous réseau et 64 bits pour l’hôte.
\end{itemize}

\subsection{Principe des options}

De nombreuses choses sont désormais en options pour l’en-tête d’un paquet ce qui permet d’alléger de manière générale la taille d’un paquet tout en permettant une plus grande flexibilité et évolutivité par le biais des options. Les options de l’en-tête sont stockées sous la forme d’une liste chaînée.

\section{Attribution des adresses en IPv6}

\subsection{Auto-configuration}

Il existe différentes méthodes avec IPv6 pour l’attribution d’une adresse IP globale aux équipements.

\begin{itemize}
\item La classique attribution manuelle\item L’auto-configuration sans état, en lien avec le NDP. Il n’y a pas besoin de serveur pour distribuer les adresses, cette dernière est auto-attribué grâce à NDP. On a ici un fonctionnement \emph{plug-and-play} au niveau de l’interface réseau.\item L’auto-configuration avec état, c’est-à-dire avec un serveur DHCPv6, très semblable à celle sous IPv4. On garde la main sur les adresses attribuées en donnant un pool d’adresse au serveur DHCP, permettant de centraliser les adresses attribuées. Nous ne détaillerons pas plus cela.
\end{itemize}

\subsection{Neighbor Discovery Protocol (NDP)}

Nous allons ici faire le lien entre l’adresse locale de lien et l’auto-configuration sans état.
Le NDP est une des nouveautés d’IPv6 et c’est ce qui contribue à son attrait. Ce protocole fournit des services semblables à ARP (Address Resolution Protocol), ICMP (Internet Control and Error Messages) et plus encore. Il est basé principalement sur le protocole ICMP qui a été enrichi par rapport à IPv4 : on parle d’ICMPv6.Voici les différentes fonctions que remplit ce protocole :
\begin{itemize}
\item La résolution d’adresse, semblable à ARP seulement ici le protocole utilise les messages standards d’ICMPv6.\item NUD (Neighbor Unreachability Detection) : c’est une des nouveautés d’IPv6, cela permet d’effacer des tables les équipements proches devenus inaccessibles. Il permet une plus grande adaptabilité pour changer une route au sein d’un routeur si un de ses voisins devient inaccessible.\item La configuration automatique d’un équipement grâce à la découverte de ses voisins avec par exemple :
\begin{itemize}
\item découverte des routeurs\item découverte des préfixes réseau, qui lui permet de construire son ou ses propres adresses IPv6 (la procédure sera expliquée plus loin)
\item découverte des adresses dupliquées, qui permet d’éviter que l’adresse construite automatiquement n’existe déjà sur le réseau\item découverte des paramètres du lien physique (taille du MTU, nombre de sauts max autorisés...)
\end{itemize}
\item	Indication de redirection, envoyée par un routeur s’il connaît une meilleure route en nombre de saut pour atteindre une destination.
\end{itemize}On peut donc voir que IPv6 remplit de nombreuses fonctions dont certaines sont nouvelles.

Voyons donc la procédure d’auto-configuration d’une adresse pour un nouvel équipement rattaché à un réseau :
\begin{itemize}
\item la toute première étape consiste à créer l'adresse locale de lien de l’équipement.
\item une fois l'unicité de cette adresse vérifiée, la machine est en mesure de communiquer avec les autres machines de distance 1. Les machines de distance 1 sont les machines du lien.
\item pour savoir comment sera attribuée l’adresse unicast globale (l’IP globale), la machine doit chercher à acquérir un message d'annonce d’un routeur
\item s'il y a un routeur sur le lien, la machine doit appliquer la méthode indiquée par le message d'annonce du routeur :
\begin{itemize}
\item l'auto-configuration sans état, la machine utilise sont adresse de lien locale comme adresse unicast globale
\item l'auto-configuration avec état (DHCPv6)
\end{itemize}
\item en l'absence de routeur sur le lien, la machine doit essayer d'acquérir l'adresse unicast globale par la méthode d'auto-configuration sans état. Si la tentative échoue, c'est terminé. Les communications se feront uniquement sur le lien avec l'adresse locale de lien. La machine n'a pas une adresse avec une portée qui l'autorise à communiquer avec des machines autres que celles du lien.
\end{itemize}

\section{Mobilité IPv6 (MIPv6)}

Le fait que l’IPv6 soit un protocole construit en amont par l’IETF, certains mécanismes pour la mobilité ont directement pu être intégrés dans la conception du protocole.Voyons déjà le problème posé. Pour communiquer avec des serveurs, des connexions sont établies et elles sont liées à l’adresse IP. Le problème lié à la mobilité c’est que l’adresse IP est liée au réseau sur lequel on se trouve, dès lors que l’on va se déplacer on va changer de réseau et obtenir une nouvelle adresse IP, les connexions actives vont donc être perdues car l’adresse IP avec laquelle elles avaient établi les connexions n’est plus joignable.Un mécanisme à donc été mis en place au sein d’IPv6 pour résoudre le problème : il faut autoriser un équipement à garder son adresse IP \textbf{même s’il change de réseau}.Pour cela un équipement est attaché à un réseau, ce sera son \textbf{réseau mère}, qui lui fournira une adresse IP fixe et ce dernier permettra d’assurer la continuité des connexions établies.On parle de \emph{home address} (HoA) pour l’adresse IP fixe liée au réseau mère (\emph{home network}).De plus l’équipement aura toujours une adresse IP variable et temporaire liée à sa localisation et au réseau auquel il est directement connecté, on parle de \emph{care-of address} (CoA).La pierre angulaire de ce système est l’agent mère (HA) situé sur le réseau mère et chargé d’assurer la correspondance entre la HoA et la CoA.

Voici un petit schéma pour résumer et expliciter tout cela :

% d'où viennent ces schémas??
\scalefig{images/schema1}{.7}{}

On retrouve le nœud correspondant, qui est le serveur à joindre, le réseau mère qui fait la liaison entre les 2 adresses et le réseau étranger où se trouve physiquement notre équipement. La HoA se situe donc sur le réseau mère et la CoA se situe sur le réseau étranger.Voyons maintenant les cas d’utilisations de ce système.

\subsection{Le mobile est dans son réseau mère}

C’est le cas le plus simple, il utilise directement sa \emph{home address} et on a un routage classique pour la communication entre le mobile et le serveur.

\scalefig{images/schema2}{.7}{}

\subsection{Le mobile est dans un réseau étranger}

C’est là que cela devient intéressant, le mobile aura donc sa HoA et une adresse étrangère CoA acquise par les mécanismes d’auto-configuration sur le réseau étranger où est physiquement le mobile. L’agent mère va donc tenir sa table d’association et associer la HoA avec la CoA. Si le serveur veut envoyer un paquet au mobile il enverra un paquet avec comme source la sienne et comme destination la \emph{home address} du mobile. Une fois arrivé à l’agent mère le paquet va être routé vers la CoA du mobile. Pour cela l’agent mère encapsule le paquet qui a donc comme source l’adresse de l’agent mère et comme destination la CoA du mobile. Une fois arrivé au mobile le paquet est désencapsulé et transmis aux couches supérieurs avec donc comme source et destinations celles du premier paquet. Au yeux du mobile c’est comme s’il était sur son réseau mère, les connexions ne sont donc plus coupées car même si la CoA change, au yeux du mobile la destination est toujours sa HoA qui ne change pas. L’encapsulation se fait grâce à l’extension d’en-tête IP-IP d’IPv6, dès lors les paquets sont protégés par IPsec et tunnelé vers la CoA en toute sécurité.

\scalefig{images/schema3}{.7}{}
\scalefig{images/schema4}{.7}{}

\subsection{Mécanisme d’association}

Pour créer ou mettre à jour une association au niveau de l’agent mère, lors d’un changement de réseau du mobile par exemple, un mécanisme spécifique de Mobilité IPv6 est utilisé, il s’agit du Binding Update (BU) appelé en français mise à jour d’association. Ce mécanisme utilise un mécanisme du NDP permettant de savoir qu’un hôte n’est plus accessible (a approfondir)

\scalefig{images/schema5}{.7}{}

Un soucis que pose ce mécanisme est une certaines inefficacité et une lenteur au niveau du routage car il faut obligatoirement passé par le réseau mère même si le réseau avec lequel on communique est proche de nous.Un mécanisme d’optimisation du routage à donc été mis en place mais il n’est pas toujours supporté par le serveur avec lequel on communique. Il faut donc garder à l’esprit que la mobilité se fait bien sans cela, il ne s’agit que d’une amélioration, pas d’une obligation.


\subsection{Optimisation du routage}

Pour gagner en efficacité dans certains cas, l’optimisation de routage a été mise en place. Le principe est simple et reprend le mécanisme général avec le réseau mère seulement ici le noeud correspondant va aussi faire office d’agent mère car il va tenir une table d’association HoA/CoA. Ainsi par ce biais la on garde le même mécanisme mais l’agent mère devient le noeud correspondant, le mobile envoie donc directement au noeud correspondant (de CoA => CN en surcouche à HoA => CN) qui va enlever l’en-tête et obtenir un paquet allant de HoA à CN, la connexion n’est donc pas brisée.

\scalefig{images/schema6}{.7}{}
\scalefig{images/schema7}{.7}{}

Il faut aussi dans ce cas faire des Binding Update pour garder à jour les tables d’associations. Le mobile doit donc savoir les agents à qui il doit envoyer ces mises à jour, il a donc une table des agents et lors d’un BU il enverra à son agent mère et au noeud correspondant supportant l’optimisation de routage. Il doit toujours tenir à jour l’agent mère car certaines connexions ne supportent pas l’optimisation suivant le serveur correspondant, il peut donc y avoir à la fois des connexions optimisées et d’autres non optimisées.

\scalefig{images/schema8}{.7}{}

\end{itemize}

\cleardoublepage

\section{Conclusion}

%todo

\cleardoublepage
\listoffigures
\end{document}
