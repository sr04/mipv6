\section{Mise en œuvre d'IPv6}

Afin d'illustrer ce projet par une démonstration technique, un routeur \emph{Cisco 1811} nous a été prêté. L'objectif est de créer deux réseaux et de faire basculer un client d'un réseau à un autre, tout en restant joignable sur son adresse IPv6 d'origine.

\subsection{Installation}

Afin de permettre à tous de travailler sur le routeur, il a été installé de manière fixe chez Émilien et relié à une \emph{Freebox v6}.
Cette Freebox dispose d'une connexion IPv6 par défaut et chaque abonné dispose de 8 sous-réseaux \texttt{/64}.

Les branchements sont les suivants :
\begin{center}
    \begin{tabular}{|l|l|}
    \hline
    Port                   & Connecté à \\ \hline
    FastEthernet 0 (WAN)   & Freebox    \\ \hline
    FastEthernet 1 (WAN)   & n.c.       \\ \hline
    FastEthernet 2-8 (LAN) & Clients    \\ \hline
    \end{tabular}
\end{center}

On commence la configuration du routeur en se connectant sur le port console. On commence donc par lui attribuer un nom et un domaine :

\begin{lstlisting}
router(config)#hostname router
router(config)#ip domain-name emilienkenler.net
\end{lstlisting}

On peut ensuite activer le SSH :

\begin{lstlisting}
router(config)#ip ssh rsa keypair-name sshkey
router(config)#crypto key generate rsa usage-keys label sshkeys \
               modulus 2048
router(config)#ip ssh version 1
router(config)#ip ssh logging events
router(config)#ip ssh time-out 60
router(config)#ip ssh authentication-retries 5
router(config)#service password-encryption
router(config)#enable secret sr04
router(config)#username sr04 password 0 sr04
router(config)#line vty 0 4
router(config-line)#login local
router(config-line)#transport input ssh
\end{lstlisting}

On définit ensuite la Freebox comme route par défaut en IPv4 :
\begin{lstlisting}
router(config)#ip route 0.0.0.0 0.0.0.0 192.168.1.254
\end{lstlisting}

On active enfin l’IPv6 sur l’interface raccordée à la Freebox ainsi que l’auto-configuration, dans le but de récupérer une adresse IPv6 dans le réseau géré par la Freebox :

\begin{lstlisting}
router(config)#ipv6 unicast-routing 
router(config)#int FastEthernet 0
router(config-if)#ipv6 enable
router(config-if)#ipv6 address autoconfig default
router#show ipv6 route
S   ::/0 [2/0]
     via FE80::224:D4FF:FE9A:18E0, FastEthernet0
\end{lstlisting}

On peut ensuite récupérer l'adresse IPv6 utilisée par le routeur :
\begin{lstlisting}
router#sh ipv6 interface FastEthernet 0
  IPv6 is enabled, link-local address is FE80::5A8D:9FF:FE14:8900 
  No Virtual link-local address(es):
  Stateless address autoconfig enabled
  Global unicast address(es):
    2A01:E35:8A38:6870:5A8D:9FF:FE14:8900, subnet is 2A01:E35:8A38:6870::/64 [EUI/CAL/PRE]
      valid lifetime 86004 preferred lifetime 86004
    2A01:E35:8A38:6875:5A8D:9FF:FE14:8900, subnet is 2A01:E35:8A38:6875::/64 [EUI]
  Joined group address(es):
    FF02::1
    FF02::2
    FF02::1:FF14:8900
  MTU is 1500 bytes
  ICMP error messages limited to one every 100 milliseconds
  ICMP redirects are enabled
  ICMP unreachables are sent
  ND DAD is enabled, number of DAD attempts: 1
  ND reachable time is 30000 milliseconds (using 18502)
  ND advertised reachable time is 0 (unspecified)
  ND advertised retransmit interval is 0 (unspecified)
  ND router advertisements are sent every 200 seconds
  ND router advertisements live for 1800 seconds
  ND advertised default router preference is Medium
  Hosts use stateless autoconfig for addresses.
\end{lstlisting}

L'adresse du routeur est donc \texttt{2a01:e35:8a38:6870:5a8d:9ff:fe14:8900}.
Pour éviter de devoir taper cette adresse à chaque connexion, on créé une entrée DNS\footnote{\emph{Domain Name System}} de type \texttt{AAAA} avec le nom \texttt{router.emilienkenler.net} :

\begin{lstlisting}
router.emilienkenler.net. 1651  IN      AAAA    2a01:e35:8a38:6870:5a8d:9ff:fe14:8900
\end{lstlisting}

Au sein de l'interface de la \emph{Freebox v6}, il est possible de demander le routage de certains réseau vers une adresse qui se trouve derrière la \emph{Freebox}.
Le routeur est alors appelé le \emph{Next Hop} : c'est le point de passage suivant pour tous les paquets adressés à ce sous-réseau.
Ceci va nous permettre d'accéder à chacun des réseaux que nous utilisons pour la démonstration depuis n'importe quel endroit sur Internet.

Dans notre cas, le réseau par défaut de la \emph{Freebox} est \texttt{2a01:e35:8a38:6870::/64}.
Dans l'interface de la Freebox, on accède donc au menu \emph{Paramètres de la freebox} > \emph{Mode avancé} > \emph{Configuration IPv6}, puis on configure les 4 sous-réseaux qui seront utilisés par le projet pour les faire pointer vers le routeur :

\begin{center}
    \begin{tabular}{|l|l|l|l|}
    \hline
    ID du sous-réseau & Adresse du sous-réseau           & Description                          & Interface \\ \hline
    2                 & \texttt{2a01:e35:8a38:6872::/64} & Réseau domicile                      & VLAN2     \\ \hline
    3                 & \texttt{2a01:e35:8a38:6873::/64} & Réseau mobile                        & VLAN3     \\ \hline
    4                 & \texttt{2a01:e35:8a38:6874::/64} & \emph{Réservé Wi-Fi}                       & -         \\ \hline
    5                 & \texttt{2a01:e35:8a38:6875::/64} & Réseau extérieur & WAN       \\ \hline
    \end{tabular}
\end{center}

% screenshot de la freebox

La configuration initiale est maintenant terminée et on peut débrancher la console.

Nous n'avons pas réussi à nous connecter avec la version 2 du protocole SSH.
La version est donc forcée à 1 dans la configuration du routeur et il faut se connecter avec la commande suivante :

\begin{lstlisting}
$ ssh -6 -1 sr04@router.emilienkenler.net
\end{lstlisting}

\subsection{Auto-configuration IPv6}

On utilise le sous-réseau \texttt{2a01:e35:8a38:6875::/64} pour tester la mise en place de l'auto-configuration au niveau du routeur.
Ce réseau est configuré sur l'interface WAN du routeur.
Comme ce sous-réseau a été délégué au routeur dans la configuration de la Freebox, l'auto-configuration est désactivée de son côté.
En activant l'auto-configuration du côté du routeur, tout périphérique connecté du côté WAN (par l'intermédiaire du switch de la Freebox), pourra négocier une adresse IPv6 avec le routeur.

On active la configuration comme suit :

\begin{lstlisting}
router(config)#int FastEthernet 0
router(config-if)#ipv6 enable
router(config-if)#ipv6 address 2a01:e35:8a38:6875::/64 eui-64
router(config-if)#ipv6 nd prefix 2a01:e35:8a38:6875::/64
\end{lstlisting}

Pour vérifier que cela fonctionne, on peut déjà vérifier que le routeur a obtenu automatiquement une adresse sur son interface WAN :
\begin{lstlisting}
router#sh ipv6 interface FastEthernet 0
FastEthernet0 is up, line protocol is up
  IPv6 is enabled, link-local address is FE80::5A8D:9FF:FE14:8900 
  No Virtual link-local address(es):
  Global unicast address(es):
    2A01:E35:8A38:6870:5A8D:9FF:FE14:8900, subnet is 2A01:E35:8A38:6870::/64
    2A01:E35:8A38:6875:5A8D:9FF:FE14:8900, subnet is 2A01:E35:8A38:6875::/64 [EUI]
\end{lstlisting}

\subsection{Configuration des VLAN}

Afin de séparer les réseaux domicile et mobile entre les différents ports du switch, on créé différents VLAN.

Le VLAN2 est utilisé pour le réseau domicile.
On le rend accessible sur les ports 2 et 3 du switch :

\begin{lstlisting}
router(config)#vlan 2
router(config-vlan)#name VLAN2

router(config)#interface FastEthernet 2
router(config-if)#switchport mode access
router(config-if)#switchport access vlan 2

router(config)#interface fastEthernet 3
router(config-if)#switchport mode access
router(config-if)#switchport access vlan 2
\end{lstlisting}

On indique ensuite l'adresse du sous-réseau au routeur et on active l'auto-configuration :

\begin{lstlisting}
router(config)#interface vlan 2
router(config-if)#ipv6 enable                              
router(config-if)#ipv6 address 2a01:e35:8a38:6872::/64 eui-64
router(config-if)#ipv6 nd ra interval 20
router(config-if)#ipv6 nd advertisement-interval
router(config-if)#ipv6 nd prefix 2a01:e35:8a38:6872::/64
\end{lstlisting}

La même configuration est appliquée pour le VLAN3, qui est utilisé pour le réseau mobile sur les ports 4 et 5 :

\begin{lstlisting}
router(config)#vlan 3
router(config-vlan)#name VLAN3

router(config)#interface FastEthernet 4
router(config-if)#switchport mode access
router(config-if)#switchport access vlan 3

router(config)#interface FastEthernet 5
router(config-if)#switchport mode access
router(config-if)#switchport access vlan 3

router(config)#interface vlan 3
router(config-if)#ipv6 enable       
router(config-if)#ipv6 address 2a01:e35:8a38:6873::/64 eui-64
router(config-if)#ipv6 nd ra interval 20
router(config-if)#ipv6 nd advertisement-interval
router(config-if)#ipv6 nd prefix 2a01:e35:8a38:6873::/64
\end{lstlisting}

\subsection{Configuration du Wi-Fi}

Nous souhaitions initialement réaliser une démonstration consistant à passer un client d'une connexion filaire à une connexion Wi-Fi.
La conservation de la même IP devrait alors permettre de faire cette bascule sans interruption des connexions.

Cependant, l'interface Wi-Fi du routeur dont nous disposons fonctionne en mode bridge avec une autre interface et les bridge ne supportent pas l'IPv6.
Il n'est donc pas possible de faire de l'IPv6 en Wi-Fi avec ce routeur.

La configuration du Wi-Fi été abandonnée.
La configuration du Wi-Fi en IPv4 est disponible en annexe à titre de référence.

\subsection{Configuration de Mobile IPv6 sur le routeur}

Afin que les clients puissent utiliser Mobile IPv6 sur les réseaux VLAN2 et VLAN3, on configure le routeur comme un \emph{Home Agent} :

\begin{lstlisting}
router(config)#interface vlan 2
router(config-if)#ipv6 mobile home-agent
router(config)#interface vlan 3
router(config-if)#ipv6 mobile home-agent
\end{lstlisting}

On peut alors afficher des détails sur l'état de cette fonctionnalité :

\begin{itemize}

\item les informations générales :
\begin{lstlisting}
router#sh ipv6 mobile globals
Mobile IPv6 Global Settings:

  3 Home Agent service on following interfaces:
    FastEthernet0
    Vlan2
    Vlan3
  Features:
    Auth-option support disabled
  Bindings:
    Maximum number is unlimited.
    0 bindings are in use
    1 bindings peak
    Binding lifetime permitted is 262140 seconds
    Recommended refresh time is 300 
\end{lstlisting}

\item les détails relatifs à chaque interface :
\begin{lstlisting}
router#sh ipv6 mobile home-agent
Home Agent information for Vlan2
  Configured:
    FE80::5A8D:9FF:FE14:8900
    preference 0 lifetime 1800
      global address 2A01:E35:8A38:6872:5A8D:9FF:FE14:8900/64

  No Discovered Home Agents
Home Agent information for Vlan3
  Configured:
    FE80::5A8D:9FF:FE14:8900
    preference 0 lifetime 1800
      global address 2A01:E35:8A38:6873:5A8D:9FF:FE14:8900/64

  No Discovered Home Agents
\end{lstlisting}

\item la liste des machines connectées
\begin{lstlisting}
router#sh ipv6 mobile binding 
Mobile IPv6 Binding Cache Entries:

Selection matched 0 bindings
\end{lstlisting}

\item des détails sur le traffic
\begin{lstlisting}
router#sh ipv6 mobile traffic
MIPv6 statistics:
  Rcvd:  111 total
      0 truncated, 0 format errors
      0 checksum errors
    Binding Updates received: 109
      0 no HA option, 0 BU's length
      0 options' length, 0 invalid CoA
  Sent:  174 generated
    Binding Acknowledgements sent: 109
      107 accepted (0 prefix discovery required)
      0 reason unspecified, 0 admin prohibited
      0 insufficient resources, 0 home reg not supported
      0 not home subnet, 2 not home agent for node
      0 DAD failed, 0 sequence number
    Binding Errors sent: 65
      63 no binding, 2 unknown MH

  Home Agent Traffic:
    98 registrations, 9 deregistrations
    2w1d since last accepted HA registration
    2w1d since last failed HA registration
    133 last failed registration code
    Traffic forwarded:
      0 tunneled, 0 reversed tunneled
    Dynamic Home Agent Address Discovery:
      0 requests received, 0 replies sent
    Mobile Prefix Discovery:
      2036 solicitations received, 2036 advertisements sent\end{lstlisting}
\end{itemize}

En cas de besoin, on peut activer l'affichage de messages supplémentaires dans les journaux relatifs à cette fonctionnalité :

\begin{lstlisting}
router#debug ipv6 mobile AR-selection
router#debug ipv6 mobile binding-cache
router#debug ipv6 mobile forwarding 
router#debug ipv6 mobile home-agent   
router#debug ipv6 mobile networks    
router#debug ipv6 mobile registrations      
router#debug ipv6 mobile router 
\end{lstlisting}

\subsection{Configuration du client}

On souhaite configurer le client en tant que \emph{Mobile Node}.

Il existe plusieurs implémentations d'IPv6 sous Linux.
Nous avons choisi \emph{UMIP} \footnote{Universal Mobile IP for Linux, \url{http://www.umip.org/}}, qui est une implementation de la RFC6275\footnote{\url{http://tools.ietf.org/html/rfc6275}}.
Une autre alternative est \emph{MIPL} \footnote{Mobile IPv6 for Linux}, dont une documentation est encore disponible \footnote{\url{http://tldp.org/HOWTO/Mobile-IPv6-HOWTO/mipv6.html}}, mais dont le site n'est plus accessible.

UMIP nécessite le support du noyau pour pouvoir fonctionner. Sous Fedora, ce support est disponible nativement et le paquet \texttt{mipv6-daemon} est disponible pour l'exploiter.

\begin{lstlisting}
# yum install mipv6-daemon
\end{lstlisting}

On peut alors créer le fichier de configuration \texttt{/etc/mip6d.conf}, qui spécifie que le démon doit agir en tant que \emph{Mobile Node} (MN), ainsi que l'adresse du réseau domicile et du \emph{Home Agent} (c'est-à-dire le routeur) :

\begin{lstlisting}
NodeConfig MN;
DebugLevel 10;
OptimisticHandoff enabled;
DoRouteOptimizationMN disabled;
MnMaxHaBindingLife 60;

Interface "em1" {
  MnIfPreference 1;
}

MnHomeLink "em1" {
  HomeAgentAddress 2A01:E35:8A38:6872:5A8D:9FF:FE14:8900;
  HomeAddress 2A01:E35:8A38:6872::1/64;
}
\end{lstlisting}

On peut ensuite redémarrer le démon pour prendre en compte cette configuration :

\begin{lstlisting}
# systemctl restart mip6d.service
\end{lstlisting}

\subsection{Analyse du traffic}

Maintenant que tout est configuré, on peut réaliser la démonstration.
L'objectif est de connecter un client dans le réseau domicile, puis de le connecter dans le réseau mobile et de vérifier qu'il est toujours joignable sur son adresse d'origine.
Pour vérifier cela, on utilisera un tiers qui envoie des \texttt{ping} vers l'adresse dans le réseau domicile.
Le tiers pourra se trouver dans un réseau quelconque.

\begin{warning}
Une première remarque importante à faire concerne la gestion de l'état des interfaces VLAN par le routeur.

Conformément à une note sur le site de Cisco\footnote{\url{https://supportforums.cisco.com/docs/DOC-2032}}, une interface VLAN n'est disponible que si au moins un périphérique est connecté à une interface dans ce VLAN.

Ainsi, si le client mobile de test est le seul périphérique connecté au switch et qu'on le branche sur un port du VLAN mobile, les ports 2 et 3 ne sont pas connectés.
Le switch va alors désactiver le VLAN2 et passer l'interface à l'état \texttt{up/down} au lieu de \texttt{up/up}.

Si on regarde l'état du Mobile IPv6 sur le routeur dans ces conditions, on constate effectivement qu'aucune adresse n'est détectée sur l'interface VLAN2 :
\begin{lstlisting}
router#sh ipv6 mobile home-agent
Home Agent information for Vlan2
  Configured:
    FE80::5A8D:9FF:FE14:8900
    preference 0 lifetime 1800
      No global addresses within prefix

  No Discovered Home Agents
Home Agent information for Vlan3
  Configured:
    FE80::5A8D:9FF:FE14:8900
    preference 0 lifetime 1800
      global address 2A01:E35:8A38:6873:5A8D:9FF:FE14:8900/64

  No Discovered Home Agents
\end{lstlisting}

Une alternative consiste à utiliser le VLAN dans lequel se trouve la Freebox, mais celle-ci va alors répondre aux trames liées à Mobile IPv6, ce qui perturbe le client.

Il faut donc veiller à laisser un périphérique connecté en permanence dans le réseau domicile.
\end{warning}

% TODO Descriptif de la capture

\subsection{Sécurisation}
