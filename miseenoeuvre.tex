\section{Mise en œuvre d'IPv6}

Afin d'illustrer ce projet par une démonstration technique, un routeur \emph{Cisco 1811} nous a été prêté. L'objectif est de créer deux réseaux et de faire basculer un client d'un réseau à un autre, tout en restant joignable sur son adresse IPv6 d'origine.

\subsection{Installation}

Afin de permettre à tous de travailler sur le routeur, le routeur a été installé de manière fixe derrière une \emph{Freebox v6}.
Cette Freebox dispose d'une connexion IPv6 par défaut et chaque abonné dispose de 8 sous-réseaux \texttt{/64}.

Les branchements sont les suivants :
\begin{center}
    \begin{tabular}{|l|l|}
    \hline
    Port                   & Connecté à \\ \hline
    FastEthernet 0 (WAN)   & Freebox    \\ \hline
    FastEthernet 1 (WAN)   & n.c.       \\ \hline
    FastEthernet 2-3 (LAN) & Clients (réseau domicile)   \\ \hline
    FastEthernet 4-5 (LAN) & Clients (réseau mobile)   \\ \hline
    FastEthernet 6-9 (LAN) & n.c.   \\ \hline
    \end{tabular}
\end{center}

Pour commencer, il faut se connecter au routeur à l'aide du port console.
La première étape consiste à identifier le routeur en lui donnant un nom :

\begin{code}
router(config)#hostname router
router(config)#ip domain-name emilienkenler.net
\end{code}

Il est alors d'activer le SSH pour pouvoir se connecter à distance :

\begin{code}
router(config)#ip ssh rsa keypair-name sshkey
router(config)#crypto key generate rsa usage-keys label sshkeys \
               modulus 2048
router(config)#ip ssh version 1
router(config)#ip ssh logging events
router(config)#ip ssh time-out 60
router(config)#ip ssh authentication-retries 5
router(config)#service password-encryption
router(config)#enable secret sr04
router(config)#username sr04 password 0 sr04
router(config)#line vty 0 4
router(config-line)#login local
router(config-line)#transport input ssh
\end{code}

\begin{warning}
Le protocole SSH ne semble pas fonctionner en version 2 : le client affiche \texttt{Invalid modulus length}.
La version 1 est donc forcée.
\end{warning}

Pour que les clients puissent joindre Internet, il faut définir la route par défaut en IPv4, qui est la Freebox :
\begin{code}
router(config)#ip route 0.0.0.0 0.0.0.0 192.168.1.254
\end{code}

Il est maintenant possible d'activer l’IPv6 et l’auto-configuration sur l’interface raccordée à la Freebox, dans le but de récupérer une adresse dans le réseau géré par la Freebox :

\begin{code}
router(config)#ipv6 unicast-routing 
router(config)#int FastEthernet 0
router(config-if)#ipv6 enable
router(config-if)#ipv6 address autoconfig default
router#show ipv6 route
S   ::/0 [2/0]
     via FE80::224:D4FF:FE9A:18E0, FastEthernet0
\end{code}

La commande suivante permet de vérifier qu'une adresse a bien été assignée :
\begin{code}
router#sh ipv6 interface FastEthernet 0
  IPv6 is enabled, link-local address is FE80::5A8D:9FF:FE14:8900 
  No Virtual link-local address(es):
  Stateless address autoconfig enabled
  Global unicast address(es):
    2A01:E35:8A38:6870:5A8D:9FF:FE14:8900, subnet is 2A01:E35:8A38:6870::/64 [EUI/CAL/PRE]
      valid lifetime 86004 preferred lifetime 86004
  Joined group address(es):
    FF02::1
    FF02::2
    FF02::1:FF14:8900
  MTU is 1500 bytes
  ICMP error messages limited to one every 100 milliseconds
  ICMP redirects are enabled
  ICMP unreachables are sent
  ND DAD is enabled, number of DAD attempts: 1
  ND reachable time is 30000 milliseconds (using 18502)
  ND advertised reachable time is 0 (unspecified)
  ND advertised retransmit interval is 0 (unspecified)
  ND router advertisements are sent every 200 seconds
  ND router advertisements live for 1800 seconds
  ND advertised default router preference is Medium
  Hosts use stateless autoconfig for addresses.
\end{code}

L'adresse du routeur est donc \texttt{2a01:e35:8a38:6870:5a8d:9ff:fe14:8900}.
Pour éviter de devoir taper cette adresse à chaque connexion, nous avons créé une entrée DNS\footnote{\emph{Domain Name System}} de type \texttt{AAAA} avec le nom \texttt{router.emilienkenler.net} :

\begin{code}
router.emilienkenler.net. 1651  IN      AAAA    2a01:e35:8a38:6870:5a8d:9ff:fe14:8900
\end{code}

Au sein de l'interface de la \emph{Freebox v6}, il est possible de demander le routage de certains réseau vers une adresse qui se trouve derrière la \emph{Freebox}.
Le routeur est alors appelé le \emph{Next Hop} : c'est le point de passage suivant pour tous les paquets adressés à ce sous-réseau.
Ceci va nous permettre d'accéder à chacun des réseaux que nous utilisons pour la démonstration depuis Internet.

Dans notre cas, le réseau par défaut de la \emph{Freebox} est \texttt{2a01:e35:8a38:6870::/64}.
Dans l'interface de la Freebox, on accède donc au menu \emph{Paramètres de la Freebox > Mode avancé > Configuration IPv6}, puis on configure les 4 sous-réseaux qui seront utilisés par le projet pour les faire pointer vers le routeur :

\scalefig{images/freebox}{.7}{}

Les réseaux seront utilisés de cette manière :

\begin{center}
    \begin{tabular}{|l|l|l|l|}
    \hline
    ID du sous-réseau & Adresse du sous-réseau           & Description                          & Interface \\ \hline
    2                 & \texttt{2a01:e35:8a38:6872::/64} & Réseau domicile                      & VLAN2     \\ \hline
    3                 & \texttt{2a01:e35:8a38:6873::/64} & Réseau mobile                        & VLAN3     \\ \hline
    4                 & \texttt{2a01:e35:8a38:6874::/64} & \emph{Réservé Wi-Fi}                       & -         \\ \hline
    5                 & \texttt{2a01:e35:8a38:6875::/64} & Réseau extérieur & WAN       \\ \hline
    \end{tabular}
\end{center}

La configuration initiale est maintenant terminée et la console peut être débranchée.

Comme SSH2 ne fonctionne pas, il faut bien penser à forcer la version 1 du protocole pour se connecter au routeur :

\begin{code}
$ ssh -6 -1 sr04@router.emilienkenler.net
\end{code}

\subsection{Auto-configuration IPv6}

Nous allons utiliser le sous-réseau \texttt{2a01:e35:8a38:6875::/64} pour tester la mise en place de l'auto-configuration au niveau du routeur.

Comme il a été délégué au routeur dans la configuration de la Freebox, celle-ci ne l'annonce plus.
Nous pouvons donc configurer ce sous-réseau sur l'interface WAN puis vérifier que les clients connectés au niveau de la Freebox parviennent toujours à le rejoindre.

La configuration se fait ainsi :

\begin{code}
router(config)#int FastEthernet 0
router(config-if)#ipv6 enable
router(config-if)#ipv6 address 2a01:e35:8a38:6875::/64 eui-64
router(config-if)#ipv6 nd prefix 2a01:e35:8a38:6875::/64
\end{code}

La configuration de l'interface montre que celle-ci a déjà récupéré une adresse dans ce sous-réseau :

\begin{code}
router#sh ipv6 interface FastEthernet 0
FastEthernet0 is up, line protocol is up
  IPv6 is enabled, link-local address is FE80::5A8D:9FF:FE14:8900 
  No Virtual link-local address(es):
  Global unicast address(es):
    2A01:E35:8A38:6870:5A8D:9FF:FE14:8900, subnet is 2A01:E35:8A38:6870::/64
    2A01:E35:8A38:6875:5A8D:9FF:FE14:8900, subnet is 2A01:E35:8A38:6875::/64 [EUI]
\end{code}

Un \texttt{ping} depuis l'extérieur montre que cette adresse est bien valide :

\begin{code}
$ ping6 2A01:E35:8A38:6875:5A8D:9FF:FE14:8900
PING 2a01:e35:8a38:6875:5a8d:9ff:fe14:8900 from 2001:41d0:8:9b46::1 eth0: 56 data bytes
64 bytes from 2a01:e35:8a38:6875:5a8d:9ff:fe14:8900: icmp_seq=1 ttl=56 time=32.0 ms
\end{code}

\subsection{Configuration des VLAN}

Afin de séparer les réseaux domicile et mobile entre les différents ports du switch, nous allons utiliser deux VLAN.

Le VLAN2 est utilisé pour le réseau domicile.
Il est accessible sur les ports 2 et 3 du switch :

\begin{code}
router(config)#vlan 2
router(config-vlan)#name VLAN2

router(config)#interface FastEthernet 2
router(config-if)#switchport mode access
router(config-if)#switchport access vlan 2

router(config)#interface fastEthernet 3
router(config-if)#switchport mode access
router(config-if)#switchport access vlan 2
\end{code}

Il faut ensuite indiquer l'adresse du sous-réseau correspondant à l'interface et activer l'auto-configuration.

\begin{code}
router(config)#interface vlan 2
router(config-if)#ipv6 enable                              
router(config-if)#ipv6 address 2a01:e35:8a38:6872::/64 eui-64
router(config-if)#ipv6 nd ra interval 20
router(config-if)#ipv6 nd advertisement-interval
router(config-if)#ipv6 nd prefix 2a01:e35:8a38:6872::/64
\end{code}

Une configuration similaire est appliquée pour le VLAN3, qui est utilisé pour le réseau mobile sur les ports 4 et 5 :

\begin{code}
router(config)#vlan 3
router(config-vlan)#name VLAN3

router(config)#interface FastEthernet 4
router(config-if)#switchport mode access
router(config-if)#switchport access vlan 3

router(config)#interface FastEthernet 5
router(config-if)#switchport mode access
router(config-if)#switchport access vlan 3

router(config)#interface vlan 3
router(config-if)#ipv6 enable       
router(config-if)#ipv6 address 2a01:e35:8a38:6873::/64 eui-64
router(config-if)#ipv6 nd ra interval 20
router(config-if)#ipv6 nd advertisement-interval
router(config-if)#ipv6 nd prefix 2a01:e35:8a38:6873::/64
\end{code}

\subsection{Configuration du Wi-Fi}

Nous souhaitions initialement réaliser une démonstration consistant à passer un client d'une connexion filaire à une connexion Wi-Fi.
La conservation de la même IP devrait alors permettre de faire cette bascule sans interruption des connexions.

Cependant, l'interface Wi-Fi du routeur dont nous disposons nécessite l'utilisation d'un bridge qui ne supporte pas l'IPv6.
Il n'est donc pas possible de faire de l'IPv6 en Wi-Fi avec ce routeur.

La configuration du Wi-Fi a donc été abandonnée.
Les commandes pour IPv4 sont toutefois disponibles en annexe à titre de référence.

\subsection{Configuration de Mobile IPv6 sur le routeur}

Afin que les clients puissent utiliser Mobile IPv6 sur les VLAN, le routeur va jouer le rôle de \emph{Home Agent} :

\begin{code}
router(config)#interface vlan 2
router(config-if)#ipv6 mobile home-agent
router(config)#interface vlan 3
router(config-if)#ipv6 mobile home-agent
\end{code}

Quatre commandes sont disponibles pour afficher des détails sur Mobile IPv6 une fois qu'il est activé (elles sont assez verbeuses et les détails ne seront pas reproduits ici) :

\begin{code}
router#sh ipv6 mobile globals
Mobile IPv6 Global Settings:
...

router#sh ipv6 mobile home-agent
Home Agent information for Vlan2
...

router#sh ipv6 mobile binding 
Mobile IPv6 Binding Cache Entries:
...

router#sh ipv6 mobile traffic
MIPv6 statistics:
...
\end{code}

En cas de besoin, on peut activer l'affichage de messages supplémentaires dans les journaux :

\begin{code}
router#debug ipv6 mobile AR-selection
router#debug ipv6 mobile binding-cache
router#debug ipv6 mobile forwarding 
router#debug ipv6 mobile home-agent   
router#debug ipv6 mobile networks    
router#debug ipv6 mobile registrations      
router#debug ipv6 mobile router 
\end{code}

\subsection{Configuration du client}

Il existe plusieurs implémentations de Mobile IPv6 sous Linux.

Nous avons choisi \emph{UMIP} \footnote{Universal Mobile IP for Linux, \url{http://www.umip.org/}}, qui est une implémentation de la RFC6275\footnote{\url{http://tools.ietf.org/html/rfc6275}}.

Une autre alternative est \emph{MIPL} \footnote{Mobile IPv6 for Linux, \url{http://www.mobile-ipv6.org} (plus disponible)}, dont une documentation est encore disponible \footnote{\url{http://tldp.org/HOWTO/Mobile-IPv6-HOWTO/mipv6.html}}, mais dont le site principal, et donc les sources, n'est plus accessible.

\emph{UMIP} nécessite le support du noyau pour pouvoir fonctionner. Sous Fedora, ce support est disponible nativement et le paquet \texttt{mipv6-daemon} peut être installé pour l'installer :

\begin{code}
# yum install mipv6-daemon
\end{code}

On peut alors créer le fichier de configuration \texttt{/etc/mip6d.conf}, qui spécifie que le client joue ici le rôle de \emph{Mobile Node} (\texttt{MN}), ainsi que l'adresse du réseau domicile et du \emph{Home Agent} (c'est-à-dire le routeur) :

\begin{code}
NodeConfig MN;
DebugLevel 10;
OptimisticHandoff enabled;
DoRouteOptimizationMN disabled;
MnMaxHaBindingLife 60;

Interface "em1" {
  MnIfPreference 1;
}

MnHomeLink "em1" {
  HomeAgentAddress 2A01:E35:8A38:6872:5A8D:9FF:FE14:8900;
  HomeAddress 2A01:E35:8A38:6872::1/64;
}
\end{code}

Il faut ensuite redémarrer le démon pour prendre en compte cette configuration :

\begin{code}
# systemctl restart mip6d.service
\end{code}

\subsection{Analyse du traffic}

Maintenant que tout est configuré, on peut réaliser la démonstration.
L'objectif est de connecter un client dans le réseau domicile, puis de le connecter dans le réseau mobile et de vérifier qu'il est toujours joignable sur son adresse d'origine.
Pour vérifier cela, on utilisera un tiers qui envoie des \texttt{ping} vers l'adresse dans le réseau domicile.
Le tiers pourra se trouver dans un réseau quelconque.

Les paquets affichés dans ce document sont issus de l'analyse avec \emph{Wireshark} de la capture \texttt{capture1.pcapng}.
La première colonne de chaque ligne correspond au numéro de séquence dans cette capture.

\begin{warning}
Une première remarque importante à faire concerne la gestion de l'état des interfaces VLAN par le routeur.

Conformément à une note sur le site de Cisco\footnote{\url{https://supportforums.cisco.com/docs/DOC-2032}}, une interface VLAN n'est disponible que si au moins un périphérique est connecté à une interface dans ce VLAN.

Ainsi, si le client mobile de test est le seul périphérique connecté au switch et qu'on le branche sur un port du VLAN mobile, les ports 2 et 3 ne sont pas connectés.
Le switch va alors désactiver le VLAN2 et passer l'interface à l'état \texttt{up/down} au lieu de \texttt{up/up}.

Si on regarde l'état du Mobile IPv6 sur le routeur dans ces conditions, on constate effectivement qu'aucune adresse n'est détectée sur l'interface VLAN2 :
\begin{code}
router#sh ipv6 mobile home-agent
Home Agent information for Vlan2
  Configured:
    FE80::5A8D:9FF:FE14:8900
    preference 0 lifetime 1800
      No global addresses within prefix

  No Discovered Home Agents
Home Agent information for Vlan3
  Configured:
    FE80::5A8D:9FF:FE14:8900
    preference 0 lifetime 1800
      global address 2A01:E35:8A38:6873:5A8D:9FF:FE14:8900/64

  No Discovered Home Agents
\end{code}

Une alternative consiste à utiliser le VLAN dans lequel se trouve la Freebox, mais celle-ci va alors répondre aux trames liées à Mobile IPv6, ce qui perturbe le client.

Il faut donc veiller à laisser un périphérique connecté en permanence dans le réseau domicile.
\end{warning}

On déroule les étapes suivantes :

\begin{enumerate}
\item Activation de l'interface sur le client A
\item Ping de B vers A, puis arrêt du ping
\item A est débranché du VLAN3 et branché sur le VLAN2
\item Ping de B vers A puis arrêt du ping
\item Démarrage de \texttt{mip6d} sur le client A
\item Ping de B vers A en continu
\item A est débranché du VLAN2 et branché sur le VLAN3
\item Le ping doit toujours fonctionner
\end{enumerate}

\subsubsection{Auto-configuration}

% J'ai pas trop compris ce qu'il se passe... Je dirais que effd reçoit un Advertisement dès le début, donc il fait un Neighbor Solicitation et c'est réglé direct ?

Après activation de l'interface réseau sur le client A, celui-ci s'auto-configure en émettant un \texttt{Router Solicitation} :

\begin{paquet}
7	1.386	fe80::a00:27ff:fe9e:effd	ff02::2	ICMPv6	70	Router Solicitation from 08:00:27:9e:ef:fd
\end{paquet}

Le routeur répond avec un \texttt{Router Advertisement} :

\begin{paquet}
10	3.166	fe80::5a8d:9ff:fe14:8900	ff02::1	ICMPv6	126	Router Advertisement from 58:8d:09:14:89:00
\end{paquet}

\subsubsection{Test dans chaque réseau}

Le premier test consiste à envoyer un ping de B vers A dans le réseau mobile.

\begin{paquet}
28	74.503	2a01:e35:8a38:6872:226:b0ff:feed:e94a	2a01:e35:8a38:6873:a00:27ff:fe9e:effd	ICMPv6	70	Echo (ping) request id=0x0450, seq=2, hop limit=63 (reply in 29)
29	74.503	2a01:e35:8a38:6873:a00:27ff:fe9e:effd	2a01:e35:8a38:6872:226:b0ff:feed:e94a	ICMPv6	70	Echo (ping) reply id=0x0450, seq=2, hop limit=64 (request in 28)
\end{paquet}

L'adresse de A dans le réseau mobile est \texttt{2a01:e35:8a38:6873:a00:27ff:fe9e:effd}, ce qui est bien une adresse dans le VLAN3.
L'adresse de B est \texttt{2a01:e35:8a38:6872:226:b0ff:feed:e94a}.
Elle est dans le VLAN2 et ne changera pas au cours de la démonstration.

Nous débranchons ensuite A du VLAN3 pour le placer dans le VLAN2, attendons que son IP soit fixée et relançons un ping vers son adresse dans le réseau domicile.

\begin{paquet}
83	121.933	2a01:e35:8a38:6872:226:b0ff:feed:e94a	2a01:e35:8a38:6872:a00:27ff:fe9e:effd	ICMPv6	70	Echo (ping) request id=0x0452, seq=0, hop limit=64 (reply in 84)
84	121.933	2a01:e35:8a38:6872:a00:27ff:fe9e:effd	2a01:e35:8a38:6872:226:b0ff:feed:e94a	ICMPv6	70	Echo (ping) reply id=0x0452, seq=0, hop limit=64 (request in 83)
\end{paquet}

Cette étape préliminaire permet de vérifier que la communication fonctionne bien entre les deux réseaux avant de mettre en œuvre l'IPv6.

\subsubsection{Mobilité}

En laissant A dans le VLAN2, c'est-à-dire son réseau domicile, le démon \texttt{mip6d} est lancé :

\begin{code}
# systemctl start mip6d.service
\end{code}

Ce service envoie alors un \texttt{Binding Update} de A vers le routeur pour indiquer qu'il est dans son réseau domicile.
Le routeur répond avec un \texttt{Binding Acknowledgment}.

\begin{paquet}
364	290.533	2a01:e35:8a38:6872::1	2a01:e35:8a38:6872:5a8d:9ff:fe14:8900	MIPv6	110	Binding Update
365	290.536	2a01:e35:8a38:6872:5a8d:9ff:fe14:8900	2a01:e35:8a38:6872::1	MIPv6	94	Binding Acknowledgement
\end{paquet}

Le ping de B vers A est alors démarré.
Le ping est maintenant effectué vers l'adresse qui a été définie dans la configuration du \texttt{mip6d}, soit \texttt{	2a01:e35:8a38:6872::1}.

\begin{paquet}
437	339.593	2a01:e35:8a38:6872:226:b0ff:feed:e94a	2a01:e35:8a38:6872::1	ICMPv6	70	Echo (ping) request id=0x0457, seq=0, hop limit=64 (reply in 438)
438	339.593	2a01:e35:8a38:6872::1	2a01:e35:8a38:6872:226:b0ff:feed:e94a	ICMPv6	70	Echo (ping) reply id=0x0457, seq=0, hop limit=64 (request in 437)
\end{paquet}

Ensuite, A est débranché du VLAN2 et branché sur un port du VLAN3.

\begin{warning}
Durant la phase de découverte, les ping de B vers A ne sont pas visibles.
Ceci provient du fait que la capture est réalisée depuis A.
\end{warning}

Après quelques secondes, A avertit le routeur de son réseau domicile avec un \texttt{Binding Update} :

\begin{paquet}
485	355.962	2a01:e35:8a38:6872::1	2a01:e35:8a38:6872:5a8d:9ff:fe14:8900	MIPv6	110	Binding Update
...
    Alternate care-of address: 2a01:e35:8a38:6873:a00:27ff:fe9e:effd
\end{paquet}

Ce paquet contient l'adresse \emph{Care Of} dans le réseau mobile : \texttt{2a01:e35:8a38:6873:a00:27ff:fe9e:effd}

La réponse revient à nouveau sous forme de \texttt{Binding Acknowledgement} :

\begin{paquet}
488	356.965	2a01:e35:8a38:6872:5a8d:9ff:fe14:8900	2a01:e35:8a38:6872::1	MIPv6	94	Binding Acknowledgement
\end{paquet}

Immédiatement, les ping retransmis par le réseau domicile sont visibles, et la réponse de A est transmise :

\begin{paquet}
493	357.601	2a01:e35:8a38:6872:226:b0ff:feed:e94a	2a01:e35:8a38:6872::1	ICMPv6	110	Echo (ping) request id=0x0457, seq=18, hop limit=63 (reply in 494)
494	357.601	2a01:e35:8a38:6872::1	2a01:e35:8a38:6872:226:b0ff:feed:e94a	ICMPv6	110	Echo (ping) reply id=0x0457, seq=18, hop limit=64 (request in 493)
\end{paquet}

On notera que la valeur \texttt{hop limit} passe de $64$ à $63$, car le paquet effectue un saut supplémentaire par le routeur du réseau domicile avant d'arriver à A.

\subsection{Sécurisation}

À partir de la partie précédente, il doit être évident que l'utilisation de Mobile IPv6 doit être sécurisée.
En effet, il paraît très simple de réaliser un faux \texttt{Binding Update} pour faire rediriger tout le traffic d'une IP tierce vers une IP que l'on contrôle.

La sécurisation de Mobile IPv6 passe par l'utilisation d'IPSec, qui permet d'assurer l'authenticité et l'intégrité des paquets.
La RFC4877\footnote{\url{http://tools.ietf.org/html/rfc4877}} décrit un tel usage.

Cisco a également développé des méthodes d'authentification, et notamment un système de clé pré-partagées (SPI), qui est décrit dans la RFC4285\footnote{\url{http://tools.ietf.org/html/rfc4285}}.
Cette méthode est simple à mettre en œuvre, mais elle n'est malheureusement pas implémentée dans UMIP.

