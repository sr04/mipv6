\section{Mobilité IPv6 (MIPv6)}


Pour communiquer avec des serveurs, des connexions sont établies et elles sont liées à l’adresse IP.
Le problème lié à la mobilité est que l’adresse IP est liée au réseau sur lequel on se trouve, dès lors que l’on va se déplacer on va changer de réseau et obtenir une nouvelle adresse IP, les connexions actives vont donc être perdues car l’adresse IP avec laquelle elles avaient établi les connexions n’est plus joignable.
Un mécanisme à donc été mis en place au sein d’IPv6 pour résoudre le problème : il faut autoriser un équipement à garder son adresse IP \textbf{même s’il change de réseau}.
Pour cela un équipement est attaché à un réseau, ce sera son \textbf{réseau mère}, qui lui fournira une adresse IP fixe et ce dernier permettra d’assurer la continuité des connexions établies.
On parle de \emph{home address} (HoA) pour l’adresse IP fixe liée au réseau mère (\emph{home network}).
De plus l’équipement aura toujours une adresse IP variable et temporaire liée à sa localisation et au réseau auquel il est directement connecté, on parle de \emph{care-of address} (CoA).
La pierre angulaire de ce système est l’agent mère (HA) situé sur le réseau mère et chargé d’assurer la correspondance entre la HoA et la CoA.

L’IPv6 étant un protocole construit en amont par l’IETF, certains mécanismes pour la mobilité ont directement pu être intégrés dans la conception du protocole.

Voici un petit schéma pour résumer et expliciter tout cela :

\scalefig{images/schema1}{.7}{}

On retrouve le nœud correspondant, qui est le serveur à joindre, le réseau mère qui fait la liaison entre les 2 adresses et le réseau étranger où se trouve physiquement notre équipement.
La HoA se situe donc sur le réseau mère et la CoA se situe sur le réseau étranger.
Voyons maintenant les cas d’utilisations de ce système.

\subsection{Le noeud mobile est dans son réseau mère}

C’est le cas le plus simple, il utilise directement sa \emph{Home Address} et on a un routage classique pour la communication entre le noeud mobile et le serveur.

\scalefig{images/schema2}{.7}{}

\subsection{Le mobile est dans un réseau étranger}

C’est là que cela devient intéressant, le mobile aura donc sa HoA et une adresse étrangère CoA acquise par les mécanismes d’auto-configuration sur le réseau étranger où est physiquement le noeud mobile.
L’agent mère va donc tenir sa table d’association et associer la HoA avec la CoA.
Si un correspondant veut envoyer un paquet au noeud mobile il enverra un paquet avec comme source la sienne et comme destination la \emph{Home Address} du noeud mobile.
Une fois arrivé à l’agent mère le paquet va être routé vers la CoA du noeud mobile.
Pour cela l’agent mère encapsule le paquet qui a donc comme source l’adresse de l’agent mère et comme destination la CoA du mobile.
Une fois arrivé au mobile le paquet est désencapsulé et transmis aux couches supérieurs avec donc comme source et destinations celles du premier paquet.
Au yeux du mobile c’est comme s’il était sur son réseau mère, les connexions ne sont donc plus coupées car même si la CoA change, au yeux du mobile la destination est toujours sa HoA qui ne change pas.
L’encapsulation se fait grâce à l’extension d’en-tête IP-IP d’IPv6, dès lors les paquets sont protégés par IPsec et tunnelé vers la CoA en toute sécurité.

\scalefig{images/schema3}{.7}{}
\scalefig{images/schema4}{.7}{}

\subsection{Mécanisme d’association}

Pour créer ou mettre à jour une association au niveau de l’agent mère, lors d’un changement de réseau du noeud mobile par exemple, un mécanisme spécifique de Mobilité IPv6 est utilisé.
Il s’agit du \emph{Binding Update (BU)}, appelé en français mise à jour d’association.
Ce mécanisme utilise un mécanisme du NDP permettant de savoir qu’un hôte n’est plus accessible.

\scalefig{images/schema5}{.7}{}

Un soucis que pose ce mécanisme est une certaines inefficacité et une lenteur au niveau du routage car il faut obligatoirement passé par le réseau mère, même si le réseau avec lequel on communique est proche de nous.
Un mécanisme d’optimisation du routage a donc été mis en place, mais il n’est pas toujours supporté par le serveur avec lequel on communique.
Il faut donc garder à l’esprit que la mobilité se fait bien sans cela, il ne s’agit que d’une amélioration, pas d’une obligation.

\subsection{Optimisation du routage}

Pour gagner en efficacité dans certains cas, l’optimisation de routage a été mise en place.
Le principe est simple et reprend le mécanisme général avec le réseau mère, seulement ici le noeud correspondant va aussi tenir une table d’association HoA/CoA, comme l'agent mère.
Ainsi par ce biais-là, on garde le même mécanisme, mais le noeud correspondant se substitue à l'agent mère.
Le noeud mobile contacte directement au noeud correspondant, qui va enlever l’en-tête et obtenir un paquet venant de la HoA et à destination du CN, la connexion n’est donc pas brisée.

\scalefig{images/schema6}{.7}{}
\scalefig{images/schema7}{.7}{}

Il faut aussi faire des Binding Update pour garder à jour les tables d’associations.
Le noeud mobile doit donc connaître les agents auxquels il doit envoyer ces mises à jour.
Il a donc une table des agents et lors d’un BU il enverra à son agent mère et au noeud correspondant supportant l’optimisation de routage.
Il doit toujours tenir à jour l’agent mèrei, car certaines connexions ne supportent pas l’optimisation suivant le serveur correspondant, il peut donc y avoir à la fois des connexions optimisées et d’autres non optimisées.

\scalefig{images/schema8}{.7}{}

